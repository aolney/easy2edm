%%%% SemDial Proceedings template by Raquel Fernández, 2013.
%%%% Modified by Simon Dobnik for the proceedings of IWCS 2019.
%%%% Modified by Andrew Olney for the proceddings of EDM 2022

%TODO fancy footer https://www.overleaf.com/learn/latex/Headers_and_footers

\documentclass[letterpaper,11pt,oneside]{book} %letterpaper %a4paper
\usepackage[utf8]{inputenc} 
\usepackage[T1]{fontenc} % fonts to encode unicode
\usepackage{times}
\usepackage{pdfpages}
%\usepackage{color}
%\usepackage{calc}
\usepackage{url}
%\usepackage{xcolor}
\pagestyle{plain}

%left align toc
\usepackage{tocloft}
\setlength{\cftpartindent}{0em}
\setlength{\cftchapindent}{0em}
\setlength{\cftsecindent}{0em}
\setlength{\cftsubsecindent}{0em}
\setlength{\cftsubsubsecindent}{0em}
\setlength{\cftsecnumwidth}{0em}

\usepackage{longtable,booktabs,array}
\usepackage[colorlinks,
%%% EDIT TITLE: %%%%%%%%%%%%%%%%%%%%%%%%%%%%%%%%%%%%%%%%%%%%%%%%%%%%
            pdftitle={Proceedings of the 15th International Conference on Educational Data Mining},
            pdfauthor={International Educational Data Mining Society},
            %pdfsubject={...},
            %pdfkeywords={...}
            linkcolor=blue,
            urlcolor=blue,
           ]{hyperref}   % hyperlinked table of contents, etc.

\usepackage[bottom=1in,right=1in,left=1in,top=1in]{geometry}

% Textarea
%\setlength{\textwidth}{17.7cm}
%\setlength{\textheight}{25cm}
%\setlength{\oddsidemargin}{-0.6cm}
%\setlength{\topmargin}{-1.5cm}

\renewcommand{\baselinestretch}{1.1}
\setlength{\parindent}{0pt}
\setlength{\parskip}{5pt}

\newcommand{\putframe}{}
\newcommand{\draft}{\renewcommand{\putframe}{\noindent\vspace*{-8pt}\textcolor{red}{\hrule height 1mm}
\vfill\noindent\textcolor{red}{\hrule height 1mm}}}

% Parameters: file name, title, authors, horizontal offset, vertical offset
\newcommand{\paper}[5]{%
\cleardoublepage
\phantomsection
%\addcontentsline{toc}{section}{\hspace{-17pt}#2} 
\addcontentsline{toc}{section}{#2}
%\addcontentsline{toc}{section}{\protect\numberline{}#2} %chapter fails unless \par used below
\addtocontents{toc}{$\,$\textit{#3}\vspace{5pt}\par} %\par} 
\includepdf[pages=-,offset={#4 #5},pagecommand={\putframe}]{#1}
\cleardoublepage}

\newcommand{\goodpaper}[3]{\paper{#1}{#2}{#3}{-0mm}{-0mm}}

%pandoc markdown to latex conversion magic (for frontmatter)
\providecommand{\tightlist}{%
  \setlength{\itemsep}{0pt}\setlength{\parskip}{0pt}}
\setcounter{secnumdepth}{-\maxdimen} % remove section numbering
\setlength{\LTleft}{0pt}

%prevent longtables from being added to TOC
\newcommand{\nocontentsline}[3]{}
\newcommand{\tocless}[2]{\bgroup\let\addcontentsline=\nocontentsline#1{#2}\egroup}


%\draft


\begin{document}
\pagenumbering{roman}
%%  TITLE PAGE 
%%%%%%%%%%%%%%%%%%%%%%%%%%%%%%%%%%%%%%%%%%%%%%%%%%%%%%%%%%%%%%%%%%%%%%%%%%%%%%%%
\pdfbookmark{Proceedings of the 15th International Conference on Educational Data Mining}{title}
\thispagestyle{empty}

\begin{center}
%  \LARGE IWCS 2019 \\
  \vspace*{55mm}
    {\bf
    %\Huge
    \LARGE
    % \fontsize{38}{46}\selectfont
    Proceedings of the 15th International Conference on \\ Educational Data Mining\\
%    \hspace*{1cm}\\ \hspace*{1cm} \\
%    \hspace*{1cm} \\ \hspace*{1cm}\\
%    \hspace*{1cm}\\
    \vspace{1cm}
    %\Huge
    \Large
    Antonija Mitrovic \& Nigel Bosch (eds).
    % Proceedings of the Conference, Long Papers\\
    \vspace{3cm} %adjust height based on sponsor logo space
    \hspace*{1cm}} \\ % Full Volume
    %\vspace{75mm}
%    \vspace{43mm}
    \Large%\LARGE
    24--27 July, 2022\\
    University of Durham \\
    Durham, United Kingdom
  \end{center}
  
% INCLUDE SPONSOR LOGOS HERE.  Upload your images with a template set.
% Then, for a file called ``logo.png'', you would use a line like the
% the following:
%
% \includegraphics[width=2.5cm]{../templates/logo.png}
\vspace*{\fill}
\begin{center}
\includegraphics[width=4.5cm]{pics/duolingo.png} \quad \includegraphics[width=4.5cm]{pics/ets.png} \quad \includegraphics[width=4.5cm]{pics/durham.png}  \quad 
\includegraphics[width=4.5cm]{pics/atom.png} 
\end{center}

\clearpage

%% DETAILS 
%%%%%%%%%%%%%%%%%%%%%%%%%%%%%%%%%%%%%%%%%%%%%%%%%%%%%%%%%%%%%%%%%%%%%%%%%%%%%%%
\pdfbookmark{ISBN}{isbn}
\thispagestyle{empty}


%\vspace*{3.5in}
\large
\noindent
\copyright 2022 International Educational Data Mining Society\\

This work is licensed under the Creative Commons Attribution-NonCommercial-NoDerivatives 4.0 International License. To view a copy of this license, visit \url{http://creativecommons.org/licenses/by-nc-nd/4.0/} or send a letter to Creative Commons, PO Box 1866, Mountain View, CA 94042, USA.\\

Download copies of this and other EDM proceedings from:

\begin{tabular}{ll}
\ \ \ \ \ \ & International Educational Data Mining Society (IEDMS)\\
&{\url{https://educationaldatamining.org}}\\
\end{tabular}\\

%\vspace*{0.6in}
\vspace*{\fill}


\textit{Proceedings of the 15th International Conference on Educational Data Mining}.\\ Antonija Mitrovic \& Nigel Bosch (eds).\\ July 24--27, 2022. 
    Durham, United Kingdom.\\
\noindent ISBN TODO\\
%TODO ADD ISBN

\clearpage

\addtocontents{toc}{\vspace{10pt}\textbf{Frontmatter}\vspace{5pt}}

%%%%%%%%%%%%%%%%%%%%%%%%%%%%%%%%%%%%%%%%%%%%%%%%%%%%%%%%%%%%%%%%%%%%%%%%%%%%%%%%%%%%%%%%%%%%%%% 
\pdfbookmark{Frontmatter}{front-matter}
\phantomsection
\addcontentsline{toc}{section}{Preface} 

%\section*{Preface}

\begin{center}
  {\Large \bf Preface}
\end{center}

\vspace*{0.5cm}

%%%%%%%%%%%%%%%%%%%%%%%%%%%%%%%%%%%%%%%%%%%%%%%%%%%%%%%%%%%%%%%%%%%%%%%%

%%% INSERT YOUR INTRO HERE
% Welcome to the ACL Workshop on Unresolved Matters. We received
% 17 submissions, and due to a rigerous review process, we rejected 16. 

For this 15th iteration of the International Conference on Educational
Data Mining (EDM 2022), the conference returned to England, this time to
Durham, with an online hybrid format for virtual participation as well.
EDM is organized under the auspices of the International Educational
Data Mining Society. The conference, held July 24th through 27th, 2022,
follows fourteen previous editions (fully online in 2021 and 2020,
Montréal 2019, Buffalo 2018, Wuhan 2017, Raleigh 2016, Madrid 2015,
London 2014, Memphis 2013, Chania 2012, Eindhoven 2011, Pittsburgh 2010,
Cordoba, 2009 and Montréal 2008).

The theme of this year's conference is Inclusion, Diversity, Equity, and
Accessibility (IDEA) in EDM Research and Practice. This theme emphasizes
the importance of considering and broadening who is included -- or not
included -- in EDM, and why. Furthermore, the theme speaks to the
importance of IDEA considerations in all stages of the research process,
from participant recruitment and selection, data collection, methods,
analysis, results, to the application of research results in the future.
The conference features three invited talks: Jennifer Hill, Professor of
Applied Statistics at New York University, USA; René Kizilcec, Assistant
Professor of Information Science at Cornell University, USA; and Judy
Robertson, Professor of Digital Learning at the University of Edinburgh,
Scotland. As in the past few years of EDM, this year's conference also
includes an invited keynote talk by the 2021 winner of the EDM Test of
Time Award. The talk is delivered by Tiffany Barnes, Distinguished
Professor of Computer Science at North Carolina State University, USA.

This year's EDM conference continued the double-blind review process
that started in 2019. The program committee was once again extended,
this time using an interest survey process, to better reflect the
community presenting works and to keep the review load for each member
manageable. EDM received 90 submissions to the full papers track (10
pages), of which 26 were accepted (28.9\%), while a further 12 were
accepted as short papers (6 pages) and 14 as posters (4 pages). There
were 56 submissions to the short paper track, of which 17 were accepted
(30.4\%) and a further 20 were accepted as posters. The poster and demo
track itself accepted 10 contributions out of 20 submissions.

The EDM 2022 conference also held a Journal of Educational Data Mining
(JEDM) Track that provides researchers a venue to deliver more
substantial mature work than is possible in a conference proceeding and
to present their work to a live audience. The papers submitted to this
track followed the JEDM peer review process. Five papers were submitted
and two papers are featured in the conference's program.

The main conference invited contributions to an Industry Track in
addition to the main track. The EDM 2022 Industry Track received six
submissions of which four were accepted. The EDM conference also
continues its tradition of providing opportunities for young researchers
to present their work and receive feedback from their peers and senior
researchers. The doctoral consortium this year features nine such
presentations.

In addition to the main program, there are six workshops and tutorials:
\emph{Causal Inference in Educational Data Mining (Third Annual Half-Day
Workshop)}, \emph{6th Educational Data Mining in Computer Science
Education (CSEDM) Workshop}, \emph{FATED 2022: Fairness, Accountability,
and Transparency in Educational Data}, \emph{The Third Workshop of The
Learner Data Institute: Big Data, Research Challenges, \& Science
Convergence in Educational Data Science}, \emph{Rethinking
Accessibility: Applications in Educational Data Mining}, and
\emph{Tutorial: Using the Open Science Framework to promote Open Science
in Education Research}.

We thank the sponsors of EDM 2022 for their generous support: DuoLingo,
ETS, Durham University Department of Computer Science, and Durham
University School of Education. We are also thankful to the senior
program committee and regular program committee members and reviewers,
without whose expert input this conference would not be possible.
Finally, we thank the entire organizing team and all authors who
submitted their work to EDM 2022. And we thank EasyChair for their
infrastructural support.

\begin{longtable}[]{@{}lll@{}}
%\toprule
& & \\
%\midrule
\endhead
\emph{Antonija Mitrovic} & University of Canterbury & Program Chair \\
\emph{Nigel Bosch} & University of Illinois Urbana--Champaign & Program
Chair \\
\emph{Alexandra I. Cristea} & Durham University & General Chair \\
\emph{Chris Brown} & Durham University & General Chair \\
%\bottomrule
\end{longtable}
\vspace{.5cm}
\hfill
\begin{minipage}[t]{0.3\textwidth}
%\begin{flushright}
July 23nd, 2022\\
Durham, England, UK
%\end{flushright}
\end{minipage}



\clearpage

%%%%%%%%%%%%%%%%%%%%%%%%%%%%%%%%%%%%%%%%%%%%%%%%%%%%%%%%%%%%%%%%%%%%%%%%%%%%%%%%%%%%%%%%%%%%%%% 
%\pdfbookmark{Organizing Committee}{pc}
\phantomsection
\addcontentsline{toc}{section}{Organizing Committee} 

\begin{center}
  {\Large \bf Organizing Committee}
\end{center}

\vspace*{0.5cm}

%%%%%%%%%%%%%%%%%%%%%%%%%%%%%%%%%%%%%%%%%%%%%%%%%%%%%%%%%%%%%%%%%%%%%%%%

%\begin{description}
%% \item{\bf Organizers:}\vspace{2mm} \\
%% John Doe, Univeristy of Southern Atlantis\\
%% Jane Example, ACME Research Labs
%
%\item{\bf Organisers:}\vspace{2mm} \\
%  \emph{Local Chairs:} Stergios Chatzikyriakidis and Simon Dobnik \\
%  \emph{Program Chairs:} Stergios Chatzikyriakidis, Vera Demberg, and Simon Dobnik \\
%  \emph{Workshops Chair:} Asad Sayeed \\
%  \emph{Student Track Chairs:} Vlad Maraev and Chatrine Qwaider \\
%  \emph{Sponsorships Chair:} Staffan Larsson \\
%  
%\vspace{3mm}
%\item{\bf Program Committee:}\vspace{2mm} \\
%Lasha Abzianidze, Laura Aina, Maxime Amblard, Krasimir Angelov, Emily M. Bender, Raffaella Bernardi, Jean-Philippe   Bernardy , Rasmus Blanck, Gemma   Boleda, Alessandro   Bondielli, Lars   Borin, Johan Bos, Ellen   Breitholtz, Harry Bunt, Aljoscha  Burchardt, Nicoletta Calzolari, Emmanuele Chersoni, Philipp Cimiano, Stephen  Clark, Robin Cooper, Philippe  de Groote, Vera   Demberg, Simon  Dobnik, Devdatt   Dubhashi, Katrin  Erk, Arash   Eshghi, Raquel  Fernández, Jonathan  Ginzburg, Matthew Gotham, Eleni   Gregoromichelaki, Justyna Grudzinska, Gözde Gül Şahin, Iryna  Gurevych , Dag  Haug, Aurelie   Herbelot, Julian  Hough, Christine  Howes, Elisabetta Jezek, Richard  Johansson, Alexandre Kabbach, Lauri  Karttunen, Ruth   Kempson, Mathieu  Lafourcade, Gabriella   Lapesa, Shalom  Lappin, Staffan   Larsson, Gianluca Lebani, Kiyong  Lee, Alessandro   Lenci, Martha   Lewis, Maria Liakata, Sharid   Loáiciga, Zhaohui Luo, Moritz  Maria, Aleksandre Maskharashvili, Stephen   Mcgregor, Louise  McNally, Bruno  Mery, Mehdi  Mirzapour, Richard   Moot, Alessandro  Moschitti, Larry  Moss, Diarmuid  O Seaghdha, Sebastian   Pado, Ludovica  Pannitto, Ivandre Paraboni, Lucia C.   Passaro, Sandro   Pezzelle, Manfred Pinkal, Paul Piwek, Massimo  Poesio, Sylvain   Pogodalla, Christopher  Potts, Stephen  Pulman, Matthew   Purver, James   Pustejovsky, Alessandro   Raganato, Giulia  Rambelli, Allan   Ramsay, Aarne   Ranta, Christian  Retoré, Martin  Riedl, Roland   Roller, Mehrnoosh Sadrzadeh, Asad   Sayeed, Tatjana   Scheffler, Sabine Schulte Im Walde, Marco S. G.   Senaldi, Manfred  Stede, Matthew  Stone, Allan Third, Kees  Van  Deemter, Eva Maria  Vecchi, Carl Vogel, Ivan  Vulić, Bonnie   Webber, Roberto   Zamparelli
%
%
%\vspace{3mm}
%\item{\bf Invited Speakers:}\vspace{2mm} \\
%  % James Goodword, Academy of Hysterical Laughter
%  Mehrnoosh Sadrzadeh, Queen Mary, University of London \\
%  Ellie Pavlick,  Brown University \\
%  Raffaella Bernardi, University of Trento

% Panelists

% Invited Paper

%\end{description}

%\hypertarget{general-chairs}{%
\subsubsection{General Chairs}%\label{general-chairs}%}

\begin{itemize}
\tightlist
\item
  Alexandra I. Cristea (Durham University, UK)
\item
  Chris Brown (Durham University, UK)
\end{itemize}

%\hypertarget{program-chairs}{%
\subsubsection{Program Chairs}%\label{program-chairs}%}

\begin{itemize}
\tightlist
\item
  Antonija Mitrovic (University of Canterbury, NZ)
\item
  Nigel Bosch (University of Illinois Urbana--Champaign, US)
\end{itemize}

%\hypertarget{workshop-tutorial-chairs}{%
\subsubsection{Workshop \& Tutorial
Chairs}%\label{workshop-tutorial-chairs}%}

\begin{itemize}
\tightlist
\item
  Angela Stewart (Carnegie Mellon University, US)
\item
  Steven Bradley (Durham University, UK)
\end{itemize}

%\hypertarget{industry-track-chairs}{%
\subsubsection{Industry Track Chairs}%\label{industry-track-chairs}%}

\begin{itemize}
\tightlist
\item
  Carol Forsyth (Educational Testing Service, US)
\item
  Stephen Fancsali (Carnegie Learning, Inc., US)
\end{itemize}

%\hypertarget{doctoral-consortium-chairs}{%
\subsubsection{Doctoral Consortium
Chairs}%\label{doctoral-consortium-chairs}%}

\begin{itemize}
\tightlist
\item
  Neil Heffernan (Worcester Polytechnic Institute, US)
\item
  Craig Stewart (Durham University, UK)
\item
  Armando Toda (Durham University, UK)
\item
  Carol Forsyth (Educational Testing Service, US)
\end{itemize}

%\hypertarget{jedm-track-chairs}{%
\subsubsection{JEDM Track Chairs}%\label{jedm-track-chairs}%}

\begin{itemize}
\tightlist
\item
  Sharon Hsiao (Santa Clara University, US)
\item
  Luc Paquette (University of Illinois Urbana--Champaign, US)
\end{itemize}

%\hypertarget{poster-demo-track-chairs}{%
\subsubsection{Poster \& Demo Track
Chairs}%\label{poster-demo-track-chairs}%}

\begin{itemize}
\tightlist
\item
  Frederick Li (Durham University, UK)
\item
  Michelle P. Banawan (Arizona State University, US)
\item
  Hassan Khosravi (University of Queensland, AU)
\end{itemize}

%\hypertarget{publicationproceedings-chairs}{%
\subsubsection{Publication/Proceedings
Chairs}%\label{publicationproceedings-chairs}%}

\begin{itemize}
\tightlist
\item
  Andrew M. Olney (University of Memphis, US)
\item
  Tahir Aduragba (Durham University, UK)
\end{itemize}

%\hypertarget{accessibility-chairs}{%
\subsubsection{Accessibility Chairs}%\label{accessibility-chairs}%}

\begin{itemize}
\tightlist
\item
  JooYoung Seo (University of Illinois Urbana--Champaign, US)
\item
  Paul Salvador Inventado (California State University Fullerton, US)
\end{itemize}

%\hypertarget{diversity-equity-and-inclusion-chair}{%
\subsubsection{Diversity, Equity, and Inclusion
Chair}%\label{diversity-equity-and-inclusion-chair}%}

\begin{itemize}
\tightlist
\item
  Agathe Merceron (Beuth University of Applied Sciences, DE)
\end{itemize}

%\hypertarget{onlinehybrid-experience-chairs}{%
\subsubsection{Online/Hybrid Experience
Chairs}%\label{onlinehybrid-experience-chairs}%}

\begin{itemize}
\tightlist
\item
  LuEttaMae Lawrence (University of California Irvine, US)
\item
  Stephen Hutt (University of Pennsylvania, US)
\end{itemize}

%\hypertarget{publicity-and-sponsorship-chair}{%
\subsubsection{Publicity and Sponsorship
Chair}%\label{publicity-and-sponsorship-chair}%}

\begin{itemize}
\tightlist
\item
  Effie Law (Durham University, UK)
\end{itemize}

%\hypertarget{web-chair}{%
\subsubsection{Web Chair}%\label{web-chair}%}

\begin{itemize}
\tightlist
\item
  Lei Shi (Durham University, UK)
\end{itemize}

%\hypertarget{local-organisers}{%
\subsubsection{Local Organisers}%\label{local-organisers}%}

\begin{itemize}
\tightlist
\item
  Jim Ridgeway
\item
  Peter Tymms
\item
  Suncica Hadzidedic
\item
  Stamos Katsigiannis
\item
  Elaine Halliday
\item
  Georgina Sales
\item
  Judith Williams
\item
  Jingyun Wang
\item
  Dorothy Monekosso
\item
  Nelly Bencomo
\item
  Jindi Wang
\end{itemize}

%\hypertarget{iedms-officers}{%
\subsubsection{IEDMS Officers}%\label{iedms-officers}%}

\begin{longtable}[]{@{}lll@{}}
%\toprule
& & \\
%\midrule
\endhead
Tiffany Barnes, & President & North Carolina State University, USA \\
Mingyu Feng, & Treasurer & WestEd, USA \\
%\bottomrule
\end{longtable}

%\hypertarget{iedms-board-of-directors}{%
\subsubsection{IEDMS Board of
Directors}%\label{iedms-board-of-directors}%}

\begin{longtable}[]{@{}ll@{}}
%\toprule
& \\
%\midrule
\endhead
Rakesh Agrawal & Data Insights Laboratories, USA \\
Ryan Baker & University of Pennsylvania, USA \\
Michel Desmarais & Polytechnique Montréal, Canada \\
Neil Heffernan & Worcester Polytechnic Institute, USA \\
Kenneth Koedinger & Carnegie Mellon University, USA \\
Luc Paquette & University of Illinois Urbana--Champaign, USA \\
Anna Rafferty & Carleton College, USA \\
Mykola Pechenizkiy & Eindhoven University of Technology, Netherlands \\
Kalina Yacef & University of Sydney, Australia \\
%\bottomrule
\end{longtable}

%\hypertarget{senior-program-committee}{%

\tocless\subsection{Senior Program Committee}%\label{senior-program-committee}%}

\begin{longtable}[]{@{}ll@{}}
%\toprule
& \\
%\midrule
\endhead
Agathe Merceron & Beuth University of Applied Sciences Berlin \\
Alex Bowers & Columbia University \\
Andrew Lan & University of Massachusetts at Amherst \\
Andrew M. Olney & University of Memphis \\
Anna Rafferty & Carleton College \\
Caitlin Mills & University of New Hampshire \\
Collin Lynch & North Carolina State University \\
Cristobal Romero & Department of Computer Sciences and Numerical
Analysis \\
Dragan Gasevic & Monash University \\
Gautam Biswas & Vanderbilt University \\
Irena Koprinska & The University of Sydney \\
James Lester & North Carolina State University \\
Jesus G. Boticario & UNED \\
Jill-Jênn Vie & Inria \\
John Stamper & Carnegie Mellon University \\
Jonathan Rowe & North Carolina State University \\
José González-Brenes & Chegg \\
Justin Reich & Massachusetts Institute of Technology \\
Kasia Muldner & Carleton University \\
Kristy Elizabeth Boyer & University of Florida \\
Luc Paquette & University of Illinois Urbana--Champaign \\
Martina Rau & University of Wisconsin - Madison \\
Michel Desmarais & Polytechnique Montréal \\
Min Chi & BeiKaZhouLi \\
Mingyu Feng & WestEd \\
Neil Heffernan & Worcester Polytechnic Institute \\
Niels Pinkwart & Humboldt-Universität zu Berlin \\
Noboru Matsuda & North Carolina State University \\
Philip I. Pavlik Jr. & University of Memphis \\
Radek Pelánek & Masaryk University Brno \\
Roger Azevedo & University of Central Florida \\
Ryan Baker & University of Pennsylvania \\
Sebastián Ventura & University of Cordoba \\
Shaghayegh Sahebi & University at Albany - SUNY \\
Sidney D'Mello & University of Colorado Boulder \\
Stefan Trausan-Matu & University Politehnica of Bucharest \\
Stephan Weibelzahl & Private University of Applied Sciences Göttingen \\
Stephen Fancsali & Carnegie Learning, Inc. \\
Steven Ritter & Carnegie Learning, Inc. \\
Vanda Luengo & Sorbonne Université - LIP6 \\
Vincent Aleven & Carnegie Mellon University \\
Zach Pardos & University of California, Berkeley \\
%\bottomrule
\end{longtable}

%\hypertarget{program-committee}{%
\tocless\subsection{Program Committee}%\label{program-committee}%}

\begin{longtable}[]{@{}
  >{\raggedright\arraybackslash}p{(\columnwidth - 2\tabcolsep) * \real{0.3636}}
  >{\raggedright\arraybackslash}p{(\columnwidth - 2\tabcolsep) * \real{0.6364}}@{}}
%\toprule
\begin{minipage}[b]{\linewidth}\raggedright
\end{minipage} & \begin{minipage}[b]{\linewidth}\raggedright
\end{minipage} \\
%\midrule
\endhead
Abhinava Barthakur & University of South Australia \\
Aditi Mallavarapu & University of Illinois Chicago \\
Ahmad Mel & Ghent University \\
Ali Darvishi & The University of Queensland \\
Amal Zouaq & Polytechnique Montréal \\
Amelia Zafra Gómez & Department of Computer Sciences and Numerical
Analysis \\
Anis Bey & Université Paul Sabatier \\
Anna Finamore & Universidade Lusófona \\
Anthony F. Botelho & Worcester Polytechnic Institute \\
April Murphy & Carnegie Learning, Inc. \\
Aïcha Bakki & Le Mans University \\
Beverly Park Woolf & University of Massachusetts at Amherst \\
Bita Akram & North Carolina State University \\
Buket Doğan & Marmara Üniversitesi \\
Carol Forsyth & Educational Testing Service \\
Chris Piech & Stanford University \\
Clara Belitz & University of Illinois Urbana--Champaign \\
Claudia Antunes & Instituto Superior Técnico - Universidade de Lisboa \\
Costin Badica & University of Craiova \\
Craig Zilles & University of Illinois Urbana--Champaign \\
Cynthia D'Angelo & University of Illinois Urbana--Champaign \\
David Pritchard & Massachusetts Institute of Technology \\
Destiny Williams-Dobosz & University of Illinois Urbana--Champaign \\
Diego Zapata-Rivera & Educational Testing Service \\
Donatella Merlini & Università di Firenze \\
Ean Teng Khor & Nanyang Technological University \\
Eliana Scheihing & Universidad Austral de Chile \\
Ella Haig & School of Computing, University of Portsmouth \\
Emily Jensen & University of Colorado Boulder \\
Erik Hemberg & ALFA \\
Feifei Han & Griffith University \\
Frank Stinar & University of Illinois Urbana--Champaign \\
Giora Alexandron & Weizmann Institute of Science \\
Guanliang Chen & Monash University \\
Guojing Zhou & University of Colorado Boulder \\
Hannah Valdiviejas & University of Illinois Urbana--Champaign \\
Hassan Khosravi & The University of Queensland \\
Hatim Lahza & The University of Queensland \\
Howard Everson & SRI International \\
Ivan Luković & University of Novi Sad \\
Jeremiah Folsom-Kovarik & Soar Technology, Inc. \\
Jia Zhu & Florida International University \\
Jiangang Hao & Educational Testing Service \\
Jihyun Park & Apple, Inc. \\
Jina Kang & University of Illinois Urbana--Champaign \\
JooYoung Seo & University of Illinois Urbana--Champaign \\
Jose Azevedo & Instituto Politécnico do Porto \\
José Raúl Romero & University of Cordoba \\
Joshua Gardner & University of Washington \\
Juho Leinonen & University of Helsinki \\
Julien Broisin & University of Toulouse \\
Julio Guerra & University of Pittsburgh \\
Jun-Ming Su & National University of Tainan \\
Keith Brawner & United States Army Research Laboratory \\
Khushboo Thaker & University of Pittsburgh \\
Lan Jiang & University of Illinois Urbana--Champaign \\
Ling Tan & Australian Council for Educational Research \\
LuEttaMae Lawrence & University of California Irvine \\
Mar Perez-Sanagustin & Pontificia Universidad Católica de Chile \\
Marcus Specht & Delft University of Technology \\
Marian Cristian Mihaescu & University of Craiova \\
Martin Hlosta & Swiss Distance University of Applied Sciences \\
Matt Myers & University of Delaware \\
Mehmet Celepkolu & University of Florida \\
Michelle Banawan & Arizona State University \\
Mirko Marras & École Polytechnique Fédérale de Lausanne (EPFL) \\
Nathan Henderson & North Carolina State University \\
Nathaniel Blanchard & Colorado State University \\
Nicholas Diana & Colgate University \\
Olga C. Santos & aDeNu Research Group (UNED) \\
Patrick Donnelly & Oregon State University Cascades \\
Paul Hur & University of Illinois Urbana--Champaign \\
Paul Salvador Inventado & California State University Fullerton \\
Paul Stefan Popescu & University of Craiova \\
Paul Wang & Georgetown University \\
Paulo Carvalho & Carnegie Mellon University \\
Pedro Manuel Moreno-Marcos & Universidad Carlos III de Madrid \\
Phillip Grimaldi & Khan Academy \\
Prateek Basavaraj & University of Central Florida \\
Rémi Venant & Le Mans Université - LIUM \\
Rene Kizilcec & Cornell University \\
Renza Campagni & Università degli Studi di Firenze \\
Roger Nkambou & Université du Québec À Montréal (UQAM) \\
Scott Crossley & Georgia State University \\
Sébastien Iksal & Université du Mans \\
Sébastien Lallé & The University of British Columbia \\
Sergey Sosnovsky & Utrecht University \\
Shahab Boumi & University of Central Florida \\
Shalini Pandey & University of Minnesota \\
Shitian Shen & North Carolina State University \\
Solmaz Abdi & The University of Queensland \\
Sotiris Kotsiantis & University of Patras \\
Spyridon Doukakis & Ionian University \\
Sreecharan Sankaranarayanan & National Institute of Technology
Karnataka, Surathkal \\
Stefan Slater & University of Pennsylvania \\
Stephen Hutt & University of Pennsylvania \\
Tanja Käser & École Polytechnique Fédérale de Lausanne (EPFL) \\
Teresa Ober & University of Notre Dame \\
Thomas Price & North Carolina State University \\
Tounwendyam Frédéric Ouedraogo & Université Norbert Zongo \\
Tuyet-Trinh Vu & Hanoi University of Science and Technology \\
Vanessa Echeverria & Escuela Superior Politécnica del Litoral \\
Vasile Rus & The University of Memphis \\
Victor Menendez-Dominguez & Universidad Autónoma de Yucatán \\
Violetta Cavalli-Sforza & Al Akhawayn University, Morocco \\
Vladimir Ivančević & University of Novi Sad \\
Wenbin Zhang & Carnegie Mellon University \\
Yang Jiang & Educational Testing Service \\
Yang Shi & North Carolina State University \\
Yi-Jung Wu & University of Wisconsin - Madison \\
Yingbin Zhang & University of Illinois Urbana--Champaign \\
Yomna M.I. Hassan & Misr International University \\
Zhuqian Zhou & Teachers College, Columbia University \\
%\bottomrule
\end{longtable}



\clearpage

%%%%%%%%%%%%%%%%%%%%%%%%%%%%%%%%%%%%%%%%%%%%%%%%%%%%%%%%%%%%%%%%%%%%%%%%%%%%%%%%%%%%%%%%%%%%%%%

%\pdfbookmark{Sponsors}{sponsors}
\phantomsection
\addcontentsline{toc}{section}{Sponsors} 



\begin{center}
  {\Large \bf Sponsors}
\end{center}

\vspace*{.5cm}

%\subsubsection{Silver} %too small
\begin{center}
  {\Large \it Silver}\\
  \vspace*{0.7cm} %adjust based on white space around logo
  \includegraphics[width=3.5cm]{pics/duolingo.png}
\end{center}

\vspace*{.5cm}

\begin{center}
  {\Large \it Bronze}\\
  \vspace*{0.5cm}%adjust based on white space around logo
  \includegraphics[width=3.5cm]{pics/ets.png} \\
  \includegraphics[width=10.5cm]{pics/cs.png} \\
  \includegraphics[width=7.5cm]{pics/education.png} 
\end{center}

\begin{center}
  {\Large \it Contributors}\\
  \vspace*{0.3cm}%adjust based on white space around logo
  \includegraphics[width=3.5cm]{pics/atom.png} 
\end{center}


\clearpage

%%%%%%%%%%%%%%%%%%%%%%%%%%%%%%%%%%%%%%%%%%%%%%%%%%%%%%%%%%%%%%%%%%%%%%%%%%%%%%%%%%%%%%%%%%%%%%%

%\pdfbookmark{Best Paper Selection}{bestpaper}
\phantomsection
\addcontentsline{toc}{section}{Best Paper Selection} 

\begin{center}
  {\Large \bf Best Paper Selection}
\end{center}

\vspace*{.5cm}

The program committee chairs discussed and nominated four full papers
and four short papers for best paper and best student paper awards,
based on reviews, review scores, and meta-reviews. The papers and
reviews (both anonymous) were then sent to a best paper award committee
who ranked the papers. The highest-ranked paper was awarded the best
paper award, while the next-highest ranked paper with a student first
author was awarded the best student paper award.

\hypertarget{best-paper-committee}{%
\subsubsection{Best paper committee}\label{best-paper-committee}}

\begin{itemize}
\tightlist
\item
  Luc Paquette
\item
  Kalina Yacef
\item
  Kenneth Koedinger
\item
  Anna Rafferty
\end{itemize}

\hypertarget{best-paper-nominees}{%
\subsubsection{Best paper nominees}\label{best-paper-nominees}}

(Full) Jiayi Zhang, Juliana Ma. Alexandra L. Andres, Stephen Hutt, Ryan
S. Baker, Jaclyn Ocumpaugh, Caitlin Mills, Jamiella Brooks, Sheela
Sethuraman and Tyron Young. \emph{Detecting SMART Model Cognitive
Operations in Mathematical Problem-Solving Process}

(Full) Vinthuy Phan, Laura Wright and Bridgette Decent. \emph{Addressing
Competing Objectives in Allocating Funds to Scholarships and Need-based
Financial Aid}

(Full) Yuyang Nie, Helene Deacon, Alona Fyshe and Carrie Demmans Epp.
\emph{Predicting Reading Comprehension Scores of Elementary School
Students}

(Full) Guojing Zhou, Robert Moulder, Chen Sun and Sidney K. D'Mello.
\emph{Investigating Temporal Dynamics Underlying Successful
Collaborative Problem Solving Behaviors with Multilevel Vector
Autoregression}

(Short) Lea Cohausz. \emph{Towards Real Interpretability of Student
Success Prediction Combining Methods of XAI and Social Science}

(Short) Juan Sanguino, Ruben Manrique, Olga Mariño, Mario Linares and
Nicolas Cardozo. \emph{Log mining for course recommendation in limited
information scenarios}

(Short) Zhikai Gao, Bradley Erickson, Yiqiao Xu, Collin Lynch, Sarah
Heckman and Tiffany Barnes. \emph{Admitting you have a problem is the first
step: Modeling when and why students seek help in programming
assignments}

(Short) Anaïs Tack and Chris Piech. \emph{The AI Teacher Test: Measuring
the Pedagogical Ability of Blender and GPT-3 in Educational Dialogues}

%\pdfbookmark{Invited talks}{invited}
%
%%\section*{Invited talks}
%
%\begin{center}
%  {\Large \bf Invited Talks}
%\end{center}
%
%\vspace*{0.5cm}
%
%\textbf{Mehrnoosh Sadrzadeh: Ellipsis in Compositional Distributional Semantics}
%
%Ellipsis is a natural language phenomenon where part of a sentence is missing and its information must be recovered from its surrounding context, as in ``Cats chase dogs and so do foxes.''. Formal semantics offers different methods for resolving ellipsis and recovering the missing information, but the problem has not been considered for distributional semantics, where words have vector embeddings and combinations thereof provide embeddings for sentences. In elliptical sentences these combinations go beyond linear as copying of elided information is necessary. I will talk about recent results in our NAACL 2019 paper, joint with G. Wijnholds, where we develop different models for embedding VP-elliptical sentences using modal sub-exponential categorial grammars. We extend existing verb disambiguation and sentence similarity datasets to ones containing elliptical phrases and evaluate our models on these datasets for a variety of linear and non-linear combinations. Our results show that indeed resolving ellipsis improves the performance of vectors and tensors on these tasks and it also sheds some light on disambiguating their sloppy and strict  readings.
%
%\bigskip
%
%\textbf{Ellie Pavlick: What Should Constitute Natural Language ``understanding''?}
%
%Natural language processing has become indisputably good over the past few years. We can perform retrieval and question answering with purported super-human accuracy, and can generate full documents of text that seem good enough to pass the Turing test. In light of these successes, it is tempting to attribute the empirical performance to a deeper "understanding" of language that the models have acquired. Measuring natural language "understanding", however, is itself an unsolved research problem. In this talk, I will discuss recent work which attempts to illuminate what it is that state-of-the-art models of language are capturing. I will describe approaches which evaluate the models' inferential behaviour, as well as approaches which rely on inspecting the models' internal structure directly. I will conclude with results on human's linguistic inferences, which highlight the challenges involved with developing prescriptivist language tasks for evaluating computational models. 
%
%\bigskip
%
%\textbf{Raffaella Bernardi: Beyond Task Success: A Closer Look at Jointly Learning to See, Ask,
%and GuessWhat}
%
%The development of conversational agents that ground language into visual information is a challenging problem that requires the integration of dialogue management skills with multimodal understanding. Recently, visual dialogue settings have entered the scene of the Machine Learning and Computer Vision communities thanks to the construction of visually grounded human-human dialogue datasets against which Neural Network models (NNs) have been challenged. I will present our work on GuessWhat?! in which two NN agents interact to each other so that one of the two (the Questioner), by asking questions to the other (the Answerer), can guess which object the Answerer has in mind among all the entities in a given image (GuessWhat?!).  I will present our Questioner model: it encodes both visual and textual inputs, produces a multimodal representation, generates natural language questions, understands the Answerers' responses and guesses the object. I will compare our model's dialogues with models that exploit much more complex learning paradigms, like Reinforcement Learning, showing that more complex machine learning methods do not necessarily correspond to better dialogue quality or even better quantitative performance. The talk is based on work available at \url{https://vista-unitn-uva.github.io/}.


\clearpage

%%%%%%%%%%%%%%%%%%%%%%%%%%%%%%%%%%%%%%%%%%%%%%%%%%%%%%%%%%%%%%%%%%%%%%%%%%%%%%%%%%%%%%%%%%%%%%%


\pdfbookmark{Table of Contents}{toc}
\setlength{\parskip}{0pt}

%different strategy needed for TOC section head
%\begin{center}
%  {\Large \bf Table of Contents}
%\end{center}
%
%\vspace*{0.5cm}

%\renewcommand{\contentsname}{\mbox{}\\[-108pt]\noindent\textbf{\Large
%    Table of Contents}\\[-28pt]}
\renewcommand{\contentsname}{\mbox{}\\[-108pt]\centering\textbf{\Large
    Table of Contents}\\[-10pt]}%technically we should have more space, but it looks weird if we do
    
\cftchapnumwidth=0pt % removes hanging indentation
\tableofcontents
\cleardoublepage

\pagenumbering{arabic}
%%%%%%%%%%%%%%%%%%%%%%%%%%%%%%%%%%%%%%%%%%%%%%%%%%%%%%%%%%%%%%%%%%%%%%%%%%%%%%%%%%%%%%%%%%%%%%%
% \pdfbookmark{Invited Talks}{invited}
% \thispagestyle{empty}
% \mbox{}\vfill
% \begin{center}
% \Huge \bf Invited Talks
% \end{center}
% \mbox{}\vfill

% \clearpage

\addtocontents{toc}{\vspace{10pt}\textbf{Abstracts}\vspace{5pt}}

\pdfbookmark{Abstracts}{abstracts}
\phantomsection
\addcontentsline{toc}{section}{Keynotes} 

%\section*{Invited talks}

\begin{center}
  {\Large \bf Keynotes}
\end{center}

\vspace*{0.5cm}
 
%\hypertarget{keynotes-1}{%
%\subsection{Keynotes}\label{keynotes-1}}

\hypertarget{deep-down-everyone-wants-to-be-causal-jennifer-hill}{%
\subsubsection{Deep Down, Everyone Wants to be Causal (Jennifer
Hill)}\label{deep-down-everyone-wants-to-be-causal-jennifer-hill}}

\emph{Abstract:} Most researchers in the social, behavioral, and health
sciences are taught to be extremely cautious in making causal claims.
However, causal inference is a necessary goal in research for addressing
many of the most pressing questions around policy and practice. In the
past decade, causal methodologists have increasingly been using and
touting the benefits of more complicated machine learning algorithms to
estimate causal effects. These methods can take some of the guesswork
out of analyses, decrease the opportunity for ``p-hacking,'' and may be
better suited for more fine-tuned tasks such as identifying varying
treatment effects and generalizing results from one population to
another. However, should these more advanced methods change our
fundamental views about how difficult it is to infer causality? In this
talk, I will discuss some potential advantages and disadvantages of
using machine learning for causal inference and emphasize ways that we
can all be more transparent in our inferences and honest about their
limitations.

\hypertarget{beyond-algorithmic-fairness-in-education-equitable-and-inclusive-decision-support-systems-renuxe9-kizilcec}{%
\subsubsection{Beyond Algorithmic Fairness in Education: Equitable and
Inclusive Decision-Support Systems (René
Kizilcec)}\label{beyond-algorithmic-fairness-in-education-equitable-and-inclusive-decision-support-systems-renuxe9-kizilcec}}

\emph{Abstract:} Advancing equity and inclusion in schools and
universities has long been a priority in education research. While
data-driven predictive models could help improve social injustices in
education, many studies from other domains suggest instead that these
models tend to exacerbate existing inequities without added precautions.
A growing body of research from the educational data mining and
neighboring communities is beginning to map out where biases are likely
to occur, what contributes to them, and how to mitigate them. These
efforts to advance algorithmic fairness are an important research
direction, but it is critical to also consider how AI systems are used
in educational contexts to support decisions and judgements. In this
talk, I will survey research on algorithmic fairness and explore the
role of human factors in AI systems and their implications for advancing
equity and inclusion in education.

\hypertarget{no-data-about-me-without-me-including-learners-and-teachers-in-educational-data-mining-judy-robertson}{%
\subsubsection{No data about me without me: Including Learners and
Teachers in Educational Data Mining (Judy
Robertson)}\label{no-data-about-me-without-me-including-learners-and-teachers-in-educational-data-mining-judy-robertson}}

\emph{Abstract:} The conference theme this year emphasises the
broadening of participation and inclusion in educational data mining; in
this talk, I will discuss methodologies for including learners and
teachers throughout the research process. This involves not only
preventing harm to young learners which might result from insufficient
care when processing their data but also embracing their participation
in the design and evaluation of educational data mining technologies. I
will argue that even young learners can and should be included in the
analysis and interpretation of data which affects them. I will give
examples of a project in which children have the role of data activists,
using classroom sensor data to explore their readiness to learn.

\hypertarget{test-of-time-award-compassionate-data-driven-tutors-for-problem-solving-and-persistence-tiffany-barnes}{%
\subsubsection{Test of Time Award: Compassionate, Data-Driven Tutors for
Problem Solving and Persistence (Tiffany
Barnes)}\label{test-of-time-award-compassionate-data-driven-tutors-for-problem-solving-and-persistence-tiffany-barnes}}

\emph{Abstract:} Determining how, when, and whether to provide
personalized support is a well-known challenge called the assistance
dilemma. A core problem in solving the assistance dilemma is the need to
discover when students are unproductive so that the tutor can intervene.
This is particularly challenging for open-ended domains, even those that
are well-structured with defined principles and goals. In this talk, I
will present a set of data-driven methods to classify, predict, and
prevent unproductive problem-solving steps in the well-structured
open-ended domains of logic and programming. Our approaches leverage and
extend my work on the Hint Factory, a set of methods that to build
data-driven intelligent tutor supports using prior student solution
attempts. In logic, we devised a HelpNeed classification model that uses
prior student data to determine when students are likely to be
unproductive and need help learning optimal problem-solving strategies.
In a controlled study, we found that students receiving proactive
assistance on logic when we predicted HelpNeed were less likely to avoid
hints during training, and produced significantly shorter, more optimal
posttest solutions in less time. In a similar vein, we have devised a
new data-driven method that uses student trace logs to identify
struggling moments during a programming assignment and determine the
appropriate time for an intervention. We validated our algorithm's
classification of struggling and progressing moments with experts rating
whether they believe an intervention is needed for a sample of 20\% of
the dataset. The result shows that our automatic struggle detection
method can accurately detect struggling students with less than 2
minutes of work with 77\% accuracy. We further evaluated a sample of 86
struggling moments, finding 6 reasons that human tutors gave for
intervention from missing key components to needing confirmation and
next steps. This research provides insight into the when and why for
programming interventions. Finally, we explore the potential of what
supports data-driven tutors can provide, from progress tracking to
worked examples and encouraging messages, and their importance for
compassionately promoting persistence in problem solving.

\clearpage

%%%%%%%%%%%%%%%%%%%%%%%%%%%%%%%%%%%%%%%%%%%%%%%%%%%%%%%%%%%%%%%%%%%%%%%%%%%%%%%%%%%%%%%%%%%%%%%

%\pdfbookmark{JEDM Presentations}{jedm}
\phantomsection
\addcontentsline{toc}{section}{JEDM Presentations}


%\section*{Invited talks}

\begin{center}
  {\Large \bf JEDM Presentations}
\end{center}

\vspace*{0.5cm}
%\hypertarget{jedm-presentations-1}{%
%\subsection{JEDM Presentations}\label{jedm-presentations-1}}

\hypertarget{empirical-evaluation-of-deep-learning-models-for-knowledge-tracing-of-hyperparameters-and-metrics-on-performance-and-replicability}{%
\subsubsection{Empirical Evaluation of Deep Learning Models for
Knowledge Tracing: Of Hyperparameters and Metrics on Performance and
Replicability}\label{empirical-evaluation-of-deep-learning-models-for-knowledge-tracing-of-hyperparameters-and-metrics-on-performance-and-replicability}}

\emph{Abstract:} New knowledge tracing models are continuously being
proposed, even at a pace where state-of-the- art models cannot be
compared with each other at the time of publication. This leads to a
situation where ranking models is hard, and the underlying reasons of
models' performance -- be it architectural choices, hyperparameter
tuning, performance metrics, or data -- is often underexplored. In this
work, we review and evaluate a body of deep learning knowledge tracing
(DLKT) models with openly available and widely-used data sets, and with
a novel data set of students learning to program. The evaluated
knowledge tracing models include Vanilla-DKT, two Long Short-Term Memory
Deep Knowledge Trac- ing (LSTM-DKT) variants, two Dynamic Key-Value
Memory Network (DKVMN) variants, and Self- Attentive Knowledge Tracing
(SAKT). As baselines, we evaluate simple non-learning models, logistic
regression and Bayesian Knowledge Tracing (BKT). To evaluate how
different aspects of DLKT models influence model performance, we test
input and output layer variations found in the compared models that are
independent of the main architectures. We study maximum attempt count
options, including filtering out long attempt sequences, that have been
implicitly and explicitly used in prior studies. We contrast the
observed performance variations against variations from non-model
properties such as randomness and hardware. Performance of models is
assessed using multiple metrics, whereby we also contrast the im- pact
of the choice of metric on model performance. The key contributions of
this work are the following: Evidence that DLKT models generally
outperform more traditional models, but not necessarily by much and not
always; Evidence that even simple baselines with little to no predictive
value may outperform DLKT models, especially in terms of accuracy --
highlighting importance of selecting proper baselines for comparison;
Disambiguation of properties that lead to better performance in DLKT
models including metric choice, input and output layer variations,
common hyperparameters, random seeding and hard- ware; Discussion of
issues in replicability when evaluating DLKT models, including
discrepancies in prior reported results and methodology. Model
implementations, evaluation code, and data are published as a part of
this work.

\hypertarget{latent-skill-mining-and-labeling-from-courseware-content}{%
\subsubsection{Latent Skill Mining and Labeling from Courseware
Content}\label{latent-skill-mining-and-labeling-from-courseware-content}}

\emph{Abstract:} A model that maps the requisite skills, or knowledge
components, to the contents of an online course is necessary to
implement many adaptive learning technologies. However, developing a
skill model and tagging courseware contents with individual skills can
be expensive and error prone. We propose a technology to automatically
identify latent skills from instructional text on existing online
courseware called SMART (\textbf{S}kill \textbf{M}odel mining with
\textbf{A}utomated detection of \textbf{R}esemblance among
\textbf{T}exts). SMART is capable of mining, labeling, and mapping
skills without using an existing skill model or student learning (aka
response) data. The goal of our proposed approach is to mine latent
skills from assessment items included in existing courseware, provide
discovered skills with human-friendly labels, and map didactic paragraph
texts with skills. This way, mapping between assessment items and
paragraph texts is formed. In doing so, automated skill models produced
by SMART will reduce the workload of courseware developers while
enabling adaptive online content at the launch of the course. In our
evaluation study, we applied SMART to two existing authentic online
courses. We then compared machine-generated skill models and
human-crafted skill models in terms of the accuracy of predicting
students' learning. We also evaluated the similarity between
machine-generated and human-crafted skill models. The results show that
student models based on SMART-generated skill models were equally
predictive of students' learning as those based on human-crafted skill
models -- as validated on two OLI courses. Also, SMART can generate
skill models that are highly similar to human-crafted models as
evidenced by the normalized mutual information (NMI) values.


% \pdfbookmark{Full Papers}{papers}
% \thispagestyle{empty}
% \mbox{}\vfill
% \begin{center}
% \Huge \bf Full Papers
% \end{center}
% \mbox{}\vfill

% \clearpage

% \addtocontents{toc}{{}\\[10pt] \textbf{Full Papers}\vspace{5pt}}



% \pdfbookmark{Poster Abstracts}{posters}
% \thispagestyle{empty}
% \mbox{}\vfill
% \begin{center}
% \Huge \bf Poster Abstracts
% \end{center}
% \mbox{}\vfill

% \clearpage

% \addtocontents{toc}{{}\\[10pt] \textbf{Poster Abstracts}\vspace{5pt}}

% \goodpaper{../pdf/IWCS_2019_paper_1.pdf}{Temporal and Aspectual Entailment}%
% {Thomas Kober, Sander Bijl de Vroe and Mark Steedman}

% \goodpaper{../pdf/IWCS_2019_paper_3.pdf}{Re-Ranking Words to Improve Interpretability of Automatically Generated Topics}%
% {Areej Alokaili, Nikolaos Aletras and Mark Stevenson}

\goodpaper{../long-papers/pdf/EDM_2022_paper_7.pdf}{Insta-Reviewer: A Data-Driven Approach for Generating Instant Feedback on Students' Project Reports}%
    {Qinjin Jia, Mitchell Young, Yunkai Xiao, Jialin Cui, Chengyuan Liu, Parvez Rashid and Edward Gehringer}

\goodpaper{../long-papers/pdf/EDM_2022_paper_10.pdf}{Sparse Factor Autoencoders for Item Response Theory}%
    {Benjamin Paa\ssen, Malwina Dywel, Melanie Fleckenstein and Niels Pinkwart}

\goodpaper{../long-papers/pdf/EDM_2022_paper_12.pdf}{Exploring Common Trends in Online Educational Experiments}%
    {Ethan Prihar, Manaal Syed, Korinn Ostrow, Stacy Shaw, Adam Sales and Neil Heffernan}

\goodpaper{../long-papers/pdf/EDM_2022_paper_18.pdf}{Designing Representations for Question Sequencing using Reinforcement Learning}%
    {Aqil Zainal Azhar, Avi Segal and Kobi Gal}

\goodpaper{../long-papers/pdf/EDM_2022_paper_21.pdf}{Code-DKT: A Code-based Knowledge Tracing Model for Programming Tasks}%
    {Yang Shi, Min Chi, Tiffany Barnes and Thomas Price}

\goodpaper{../long-papers/pdf/EDM_2022_paper_22.pdf}{Building a Reinforcement Learning Environment from Limited Data to Optimize Teachable Robot Interventions}%
    {Tristan Maidment, Mingzhi Yu, Nikki Lobczowski, Adriana Kovashka, Erin Walker, Diane Litman and Timothy Nokes-Malach}

\goodpaper{../long-papers/pdf/EDM_2022_paper_35.pdf}{Detecting SMART Model Cognitive Operations in Mathematical Problem-Solving Process}%
    {Jiayi Zhang, Juliana Ma. Alexandra L. Andres, Stephen Hutt, Ryan S. Baker, Jaclyn Ocumpaugh, Caitlin Mills, Jamiella Brooks, Sheela Sethuraman and Tyron Young}

\goodpaper{../long-papers/pdf/EDM_2022_paper_37.pdf}{SQL-DP:A Novel Difficulty Prediction Framework for SQL Programming Problems}%
    {Jia Xu, Tingting Wei and Pin Lv}

\goodpaper{../long-papers/pdf/EDM_2022_paper_51.pdf}{Evaluating the Explainers: Black-Box Explainable Machine Learning for Student Success Prediction in MOOCs}%
    {Vinitra Swamy, Bahar Radmehr, Natasa Krco, Mirko Marras and Tanja K\"{a}ser}

\goodpaper{../long-papers/pdf/EDM_2022_paper_53.pdf}{Addressing Competing Objectives in Allocating Funds to Scholarships and Need-based Financial Aid}%
    {Vinthuy Phan, Laura Wright and Bridgette Decent}

\goodpaper{../long-papers/pdf/EDM_2022_paper_54.pdf}{Automatic Short Math Answer Grading via In-context Meta-learning}%
    {Mengxue Zhang, Sami Baral, Neil Heffernan and Andrew Lan}

\goodpaper{../long-papers/pdf/EDM_2022_paper_55.pdf}{Investigating Multimodal Predictors of Peer Satisfaction for Collaborative Coding in Middle School}%
    {Yingbo Ma, Gloria Ashiya Katuka, Mehmet Celepkolu and Kristy Elizabeth Boyer}

\goodpaper{../long-papers/pdf/EDM_2022_paper_56.pdf}{Going Deep and Far: Gaze-based Models Predict Multiple Depths of Comprehension During and One Week Following Reading}%
    {Megan Caruso, Candace Peacock, Rosy Southwell, Guojing Zhou and Sidney D'Mello}

\goodpaper{../long-papers/pdf/EDM_2022_paper_62.pdf}{Predicting Reading Comprehension Scores of Elementary School Students}%
    {Yuyang Nie, Helene Deacon, Alona Fyshe and Carrie Demmans Epp}

\goodpaper{../long-papers/pdf/EDM_2022_paper_63.pdf}{Enhancing Stealth Assessment in Game-Based Learning Environments with Generative Zero-Shot Learning}%
    {Nathan Henderson, Halim Acosta, Wookhee Min, Bradford Mott, Trudi Lord, Frieda Reichsman, Chad Dorsey, Eric Wiebe and James Lester}

\goodpaper{../long-papers/pdf/EDM_2022_paper_78.pdf}{Generalisable Methods for Early Prediction in Interactive Simulations for Education}%
    {Jade Ma\"{\i} Cock, Mirko Marras, Christian Giang and Tanja K\"{a}ser}

\goodpaper{../long-papers/pdf/EDM_2022_paper_80.pdf}{Toward Better Grade Prediction via A2GP - An Academic Achievement Inspired Predictive Model}%
    {Wei Qiu, S. Supraja and Andy W. H. Khong}

\goodpaper{../long-papers/pdf/EDM_2022_paper_84.pdf}{Individual Fairness Evaluation for Automated Essay Scoring System}%
    {Afrizal Doewes, Akrati Saxena, Yulong Pei and Mykola Pechenizkiy}

\goodpaper{../long-papers/pdf/EDM_2022_paper_94.pdf}{Combining domain modelling and student modelling techniques in a single automated pipeline}%
    {Gio Picones, Benjamin Paa\ssen, Irena Koprinska and Kalina Yacef}

\goodpaper{../long-papers/pdf/EDM_2022_paper_108.pdf}{Neural Recall Network: A Neural Network Solution to Low Recall Problem in Regex-based Qualitative Coding}%
    {Zhiqiang Cai, Cody Marquart and David Shaffer}

\goodpaper{../long-papers/pdf/EDM_2022_paper_109.pdf}{Towards Including Instructor Features in Student Grade Prediction}%
    {Nathan Ong, Jiaye Zhu and Daniel Mosse}

\goodpaper{../long-papers/pdf/EDM_2022_paper_125.pdf}{Item Response Theory-Based Gaming Detection}%
    {Yun Huang, Steven Dang, J. Elizabeth Richey, Michael Asher, Nikki G. Lobczowski, Danielle Chine, Elizabeth A. McLaughlin, Judith M. Harackiewicz, Vincent Aleven and Kenneth Koedinger}

\goodpaper{../long-papers/pdf/EDM_2022_paper_133.pdf}{Exploring Cultural Diversity and Collaborative Team Communication through a Dynamical Systems Lens}%
    {Mohammad Amin Samadi, Jacqueline G. Cavazos, Yiwen Lin and Nia Nixon}

\goodpaper{../long-papers/pdf/EDM_2022_paper_141.pdf}{Predicting Cognitive Engagement in Online Course Discussion Forums}%
    {Guher Gorgun, Seyma Nur Yildirim-Erbasli and Carrie Demmans Epp}

\goodpaper{../long-papers/pdf/EDM_2022_paper_148.pdf}{Investigating Temporal Dynamics Underlying Successful Collaborative Problem Solving Behaviors with Multilevel Vector Autoregression}%
    {Guojing Zhou, Robert Moulder, Chen Sun and Sidney D'Mello}

\goodpaper{../long-papers/pdf/EDM_2022_paper_155.pdf}{Challenges and Feasibility of Automatic Speech Recognition for Modeling Student Collaborative Discourse in Classrooms}%
    {Rosy Southwell, Samuel Pugh, E. Margaret Perkoff, Charis Clevenger, Jeffrey Bush, Rachel Lieber, Wayne Ward, Peter Foltz and Sidney D'Mello}

\goodpaper{../long-papers/pdf/EDM_2022_paper_3.pdf}{Does Practice Make Perfect? Analyzing the Relationship Between Higher Mastery and Forgetting in an Adaptive Learning System}%
    {Jeffrey Matayoshi, Eric Cosyn and Hasan Uzun}

\goodpaper{../short-papers/pdf/EDM_2022_paper_9.pdf}{Improving Peer Assessment with Graph Neural Networks}%
    {Alireza A. Namanloo, Julie Thorpe and Amirali Salehi-Abari}

\goodpaper{../short-papers/pdf/EDM_2022_paper_14.pdf}{Using Neural Network-Based Knowledge Tracing for a Learning System with Unreliable Skill Tags}%
    {Shamya Karumbaiah, Jiayi Zhang, Ryan Baker, Richard Scruggs, Whitney Cade, Margaret Clements and Shuqiong Lin}

\goodpaper{../long-papers/pdf/EDM_2022_paper_16.pdf}{\#lets-discuss: Analyzing Student Affect in Course Forums Using Emoji}%
    {Ariel Blobstein, Kobi Gal, David Karger, Marc Facciotti, Hyunsoo Kim, Jumana Almahmoud and Kamali Sripathi}

\goodpaper{../long-papers/pdf/EDM_2022_paper_20.pdf}{Investigating Growth of Representational Competencies by Knowledge-Component Model}%
    {Jihyun Rho, Martina Rau and Barry Vanveen}

\goodpaper{../short-papers/pdf/EDM_2022_paper_31.pdf}{Adversarial bandits for drawing generalizable conclusions in non-adversarial experiments: an empirical study}%
    {Yang Zhi-Han, Shiyue Zhang and Anna Rafferty}

\goodpaper{../short-papers/pdf/EDM_2022_paper_40.pdf}{Towards Real Interpretability of Student Success Prediction Combining Methods of XAI and Social Science}%
    {Lea Cohausz}

\goodpaper{../short-papers/pdf/EDM_2022_paper_42.pdf}{An Evaluation of code2vec Embeddings for Scratch}%
    {Benedikt Fein, Isabella Gra\ssl, Florian Beck and Gordon Fraser}

\goodpaper{../long-papers/pdf/EDM_2022_paper_43.pdf}{Investigating the effect of Automated Feedback on learning behavior in MOOCs for programming}%
    {Hagit Gabbay and Anat Cohen}

\goodpaper{../short-papers/pdf/EDM_2022_paper_45.pdf}{Data-driven goal setting: Searching optimal badges in the decision forest}%
    {Julian Langenhagen}

\goodpaper{../short-papers/pdf/EDM_2022_paper_52.pdf}{Improving problem detection in peer assessment through pseudo-labeling using semi-supervised learning}%
    {Chengyuan Liu, Jialin Cui, Ruixuan Shang, Yunkai Xiao, Qinjin Jia and Edward Gehringer}

\goodpaper{../long-papers/pdf/EDM_2022_paper_59.pdf}{Evaluating Gaming Detector Models For Robustness Over Time}%
    {Nathan Levin, Ryan Baker, Nidhi Nasiar, Stephen Fancsali and Stephen Hutt}

\goodpaper{../long-papers/pdf/EDM_2022_paper_69.pdf}{Characterizing joint attention dynamics during collaborative problem-solving in an immersive astronomy simulation}%
    {Yiqiu Zhou and Jina Kang}

\goodpaper{../long-papers/pdf/EDM_2022_paper_72.pdf}{Can Population-based Engagement improve Personalisation? A Novel Dataset, Baselines and Experiments}%
    {Sahan Bulathwela, Meghana Verma, Mar\'{\i}a P\'{e}rez Ortiz, Emine Yilmaz and John Shawe-Taylor}

\goodpaper{../long-papers/pdf/EDM_2022_paper_74.pdf}{Is there Method in Your Mistakes? Capturing Error Contexts by Graph Mining for Targeted Feedback}%
    {Maximilian Jahnke and Frank H\"{o}ppner}

\goodpaper{../short-papers/pdf/EDM_2022_paper_82.pdf}{Log mining for course recommendation in limited information scenarios}%
    {Juan Sanguino, Ruben Manrique, Olga Mari\~{n}o, Mario Linares and Nicolas Cardozo}

\goodpaper{../long-papers/pdf/EDM_2022_paper_93.pdf}{Using Machine Learning Explainability Methods to Personalize Interventions for Students}%
    {Paul Hur, HaeJin Lee, Suma Bhat and Nigel Bosch}

\goodpaper{../long-papers/pdf/EDM_2022_paper_97.pdf}{Grade Prediction via Prior Grades and Text Mining on Course Descriptions: Course Outlines and Intended Learning Outcomes}%
    {Jiawei Li, S. Supraja, Wei Qiu and Andy W. H. Khong}

\goodpaper{../long-papers/pdf/EDM_2022_paper_105.pdf}{From \lbraceSolution\rbrace Synthesis to \lbraceStudent Attempt\rbrace Synthesis for Block-Based Visual Programming Tasks}%
    {Adish Singla and Nikitas Theodoropoulos}

\goodpaper{../short-papers/pdf/EDM_2022_paper_107.pdf}{Mining Assignment Submission Time to Detect At-Risk Students with Peer Information}%
    {Yuancheng Wang, Nanyu Luo and Jianjun Zhou}

\goodpaper{../short-papers/pdf/EDM_2022_paper_118.pdf}{Simulating Policy Changes in Prerequisite-Free Curricula: A Supervised Data-Driven Approach}%
    {Frederik Baucks and Laurenz Wiskott}

\goodpaper{../short-papers/pdf/EDM_2022_paper_120.pdf}{Modeling One-on-one Online Tutoring Discourse using an Accountable Talk Framework}%
    {Renu Balyan, Tracy Arner, Karen Taylor, Jinnie Shin, Michelle Banawan, Walter Leite and Danielle McNamara}

\goodpaper{../short-papers/pdf/EDM_2022_paper_126.pdf}{Using Markov Models and Random Walks to Examine Strategy Use of More or Less Successful Comprehenders}%
    {Katerina Christhilf, Natalie Newton, Reese Butterfuss, Kathryn S. McCarthy, Laura K. Allen, Joseph P. Magliano and Danielle S. McNamara}

\goodpaper{../long-papers/pdf/EDM_2022_paper_130.pdf}{No Meaning Left Unlearned: Predicting Learners' Knowledge of Atypical Meanings of Words from Vocabulary Tests for Their Typical Meanings}%
    {Yo Ehara}

\goodpaper{../short-papers/pdf/EDM_2022_paper_145.pdf}{Using community-based problems to increase motivation in a data science virtual internship}%
    {Jillian Johnson and Andrew Olney}

\goodpaper{../short-papers/pdf/EDM_2022_paper_151.pdf}{Admitting you have a problem is the first step: Modeling when and why students seek help in programming assignments.}%
    {Zhikai Gao, Bradley Erickson, Yiqiao Xu, Collin Lynch, Sarah Heckman and Tiffany Barnes}

\goodpaper{../short-papers/pdf/EDM_2022_paper_153.pdf}{Going beyond \textquotedblleftGood job\textquotedblright: Analyzing helpful feedback from students' perspectives.}%
    {M Parvez Rashid, Yunkai Xiao and Edward F. Gehringer}

\goodpaper{../short-papers/pdf/EDM_2022_paper_168.pdf}{The AI Teacher Test: Measuring the Pedagogical Ability of Blender and GPT-3 in Educational Dialogues}%
    {Ana\"{\i}s Tack and Chris Piech}

\goodpaper{../short-papers/pdf/EDM_2022_paper_195.pdf}{Automatic Classification of Learning Objectives Based on Bloom's Taxonomy}%
    {Yuheng Li, Mladen Rakovic, Boon Xin Poh, Dragan Gasevic and Guanliang Chen}

\goodpaper{../long-papers/pdf/EDM_2022_paper_6.pdf}{Skills Taught vs Skills Sought: Using Skills Analytics to Identify the Gaps between Curriculum and Job Markets}%
    {Alireza Ahadi, Kirsty Kitto, Marian-Andrei Rizoiu and Katarzyna Musial}

\goodpaper{../long-papers/pdf/EDM_2022_paper_15.pdf}{DeepIRT with a Hypernetwork to Optimize the Degree of Forgetting of Past Data}%
    {Emiko Tsutsumi, Yiming Guo and Maomi Ueno}

\goodpaper{../short-papers/pdf/EDM_2022_paper_17.pdf}{Mining and Assessing Anomalies in Students' Online Learning Activities with Self-supervised Machine Learning}%
    {Lan Jiang and Nigel Bosch}

\goodpaper{../short-papers/pdf/EDM_2022_paper_23.pdf}{Faster Confidence Intervals for Item Response Theory via an Approximate Likelihood Profile}%
    {Benjamin Paa\ssen, Christina G\"{o}pfert and Niels Pinkwart}

\goodpaper{../long-papers/pdf/EDM_2022_paper_26.pdf}{Towards the understanding of cultural differences in between gamification preferences: A data-driven comparison between the US and Brazil}%
    {Armando Toda, Ana Klock, Filipe Dwan Pereira, Luiz Antonio Rodrigues, Paula Toledo Palomino, Vinicius Lopes, Craig Stewart, Elaine H. T. Oliveira, Isabela Gasparini, Seiji Isotani and Alexandra Cristea}

\goodpaper{../short-papers/pdf/EDM_2022_paper_33.pdf}{Supervised Machine Learning for Modelling STEM Career and Education Interest in Irish School Children}%
    {Annika Lindh, Keith Quille, Aidan Mooney, Kevin Marshall and Katriona O'Sullivan}

\goodpaper{../short-papers/pdf/EDM_2022_paper_39.pdf}{Equitable Ability Estimation in Neurodivergent Student Populations with Zero-Inflated Learner Models}%
    {Niall Twomey, Sarah McMullan, Anat Elhalal, Rafael Poyiadzi and Luis Vaquero}

\goodpaper{../short-papers/pdf/EDM_2022_paper_48.pdf}{Equity and Fairness of Bayesian Knowledge Tracing}%
    {Sebastian Tschiatschek, Maria Knobelsdorf and Adish Singla}

\goodpaper{../short-papers/pdf/EDM_2022_paper_57.pdf}{Analyzing the Equity of the Brazilian National High School Exam by Validating the Item Response Theory's Invariance}%
    {Vitoria Guardieiro, Marcos M. Raimundo and Jorge Poco}

\goodpaper{../long-papers/pdf/EDM_2022_paper_61.pdf}{A deep dive into microphone hardware for recording collaborative group work}%
    {Mariah Bradford, Paige Hansen, J. Ross Beveridge, Nikhil Krishnaswamy and Nathaniel Blanchard}

\goodpaper{../short-papers/pdf/EDM_2022_paper_67.pdf}{Linguistic Profiles, Question Framing and Performance in a Conversation-based Assessment System}%
    {Carol Forsyth, Jesse Sparks, Jonathan Steinberg and Laura McCulla}

\goodpaper{../long-papers/pdf/EDM_2022_paper_71.pdf}{Does chronology matter? Sequential vs contextual approaches to knowledge tracing}%
    {Yueqi Wang and Zachary Pardos}

\goodpaper{../short-papers/pdf/EDM_2022_paper_86.pdf}{Algorithmic unfairness mitigation in student models: When fairer methods lead to unintended results}%
    {Frank Stinar and Nigel Bosch}

\goodpaper{../short-papers/pdf/EDM_2022_paper_87.pdf}{The Impact of Semester Gaps on Student Grades}%
    {Gary Weiss, Joseph Denham and Daniel Leeds}

\goodpaper{../long-papers/pdf/EDM_2022_paper_88.pdf}{Assessing Instructor Effectiveness Based on Future Student Performance}%
    {Gary Weiss, Erik Brown, Michael Riad-Zaky, Ruby Iannone and Daniel Leeds}

\goodpaper{../short-papers/pdf/EDM_2022_paper_90.pdf}{Modeling study duration considering course enrollments and student diversity}%
    {Niels Seidel}

\goodpaper{../long-papers/pdf/EDM_2022_paper_91.pdf}{Generalized Sequential Pattern Mining of Undergraduate Courses}%
    {Daniel Leeds, Cody Chen, Yijun Zhao, Fiza Metla, James Guest and Gary Weiss}

\goodpaper{../short-papers/pdf/EDM_2022_paper_101.pdf}{Employing Tree-based Algorithms to Predict Students' Self-Efficacy in PISA 2018}%
    {Bin Tan and Maria Cutumisu}

\goodpaper{../short-papers/pdf/EDM_2022_paper_102.pdf}{Identifying Longitudinal Attendance Patterns through Student Subpopulation Distribution Comparison}%
    {Zitong Zhao, Pan Deng and Jianjun Zhou}

\goodpaper{../short-papers/pdf/EDM_2022_paper_103.pdf}{Improved Automated Essay Scoring using Gaussian Multi-Class SMOTE for Dataset Sampling}%
    {Jih Soong Tan, Ian K. T. Tan, Lay Ki Soon and Huey Fang Ong}

\goodpaper{../short-papers/pdf/EDM_2022_paper_110.pdf}{A Topic-Centric Crowdsourced Assisted Biomedical Literature Review Framework for Academics}%
    {Ryan Hodgson, Jingyun Wang, Alexandra Cristea, Fumiko Matsuzaki and Hiroyuki Kubota}

\goodpaper{../long-papers/pdf/EDM_2022_paper_115.pdf}{Personalized and Explainable Course Recommendations for Students at Risk of Dropping out}%
    {Kerstin Wagner, Agathe Merceron, Petra Sauer and Niels Pinkwart}

\goodpaper{../short-papers/pdf/EDM_2022_paper_119.pdf}{A deep reinforcement learning approach to automatic formative feedback}%
    {Aubrey Condor and Zachary Pardos}

\goodpaper{../short-papers/pdf/EDM_2022_paper_124.pdf}{Preliminary Experiments with Transformer based Approaches To Automatically Inferring Domain Models from Textbooks}%
    {Rabin Banjade, Priti Oli, Lasang Jimba Tamang and Vasile Rus}

\goodpaper{../long-papers/pdf/EDM_2022_paper_132.pdf}{E-learning Preparedness: A Key Consideration to Promote Fair Learning Analytics Development in Higher Education}%
    {Jinnie Shin, Okan Bulut and Wallace N. Pinto Jr.}

\goodpaper{../long-papers/pdf/EDM_2022_paper_136.pdf}{Leveraging Auxiliary Data from Similar Problems to Improve Automatic Open Response Scoring}%
    {Raysa Rivera-Bergollo, Sami Baral, Anthony Botelho and Neil Heffernan}

\goodpaper{../short-papers/pdf/EDM_2022_paper_138.pdf}{Modifying Deep Knowledge Tracing for Multi-step Problems}%
    {Qiao Zhang, Zeyu Chen, Natasha Lalwani and Christopher MacLellan}

\goodpaper{../long-papers/pdf/EDM_2022_paper_157.pdf}{Clustering Students Using Pre-Midterm Behaviour Data and Predict Their Exam Performance}%
    {Huanyi Chen and Paul Ward}

\goodpaper{../short-papers/pdf/EDM_2022_paper_162.pdf}{Format-Aware Item Response Theory for Predicting Vocabulary Proficiency}%
    {Boxuan Ma, Gayan Prasad Hettiarachchi and Yuji Ando}

\goodpaper{../short-papers/pdf/EDM_2022_paper_167.pdf}{Towards Generalized Methods for Automatic Question Generation in Educational Domains}%
    {Shravya Bhat, Huy Nguyen, Steven Moore, John Stamper, Majd Sakr and Eric Nyberg}

\goodpaper{../long-papers/pdf/EDM_2022_paper_175.pdf}{MOOC-Rec: Instructional Video Clip Recommendation for MOOC Forum Questions}%
    {Peide Zhu, Claudia Hauff and Jie Yang}

\goodpaper{../long-papers/pdf/EDM_2022_paper_186.pdf}{Automatical Graph-based Knowledge Tracing}%
    {Ting Long, Yunfei Liu, Weinan Zhang, Wei Xia, Zhicheng He, Ruiming Tang and Yong Yu}

\goodpaper{../long-papers/pdf/EDM_2022_paper_187.pdf}{Process-BERT: A Framework for Representation Learning on Educational Process Data}%
    {Alexander Scarlatos, Christopher Brinton and Andrew Lan}

\goodpaper{../short-papers/pdf/EDM_2022_paper_196.pdf}{Online Item Response Theory (OIRT) - Tracking Student Abilities in Online Learning System}%
    {Luyao Peng and Chengzhi Wei}

\goodpaper{../posters/pdf/EDM_2022_paper_200.pdf}{Looking for the best data fusion model in Smart Learning Environments for detecting at risk university students}%
    {Cristobal Romero, Wilson Chango and Rebeca Cerezo}

\goodpaper{../posters/pdf/EDM_2022_paper_202.pdf}{Distance measure between instructor-recommended and learner's learning pathways}%
    {Marie-Luce Bourguet and Yushan Li}

\goodpaper{../posters/pdf/EDM_2022_paper_205.pdf}{Student Perception on the Effectiveness of On-Demand Assistance in Online Learning Platforms}%
    {Aaron Haim and Neil Heffernan}

\goodpaper{../posters/pdf/EDM_2022_paper_211.pdf}{Theory-Informed Problem-Solving Sequential Pattern Vis-ualization}%
    {Zilong Pan and Min Liu}

\goodpaper{../posters/pdf/EDM_2022_paper_214.pdf}{CHUNK Learning: A Tool that Supports Personalized Education}%
    {Ralucca Gera, D'Marie Bartolf, Simona Tick and Akrati Saxena}

\goodpaper{../posters/pdf/EDM_2022_paper_215.pdf}{Do College Students Learn Math the Same Way as Middle School Students? On the Transferability of Findings on Within-Problem Supports in Intelligent Tutoring Systems}%
    {Michael Smalenberger and Kelly Smalenberger}

\goodpaper{../posters/pdf/EDM_2022_paper_220.pdf}{Comparison of Learning Behaviors on an e-Book System in 2019 Onsite and 2020 Online Courses}%
    {Hiroaki Kawashima}

\goodpaper{../posters/pdf/EDM_2022_paper_221.pdf}{Recommendation System of Mobile Language Learning Applications: Similarity versus Diversity in Learner Preference}%
    {Juyeong Song, Kisu Yang, Hyeji Jang and Hyo-Jeong So}

\goodpaper{../posters/pdf/EDM_2022_paper_222.pdf}{A Variant of Performance Factors Analysis Model for Categorization}%
    {Meng Cao and Philip Pavlik}

\goodpaper{../posters/pdf/EDM_2022_paper_223.pdf}{Selecting Reading Texts Suitable for Incidental Vocabulary Learning by Considering the Estimated Distribution of Acquired Vocabulary}%
    {Yo Ehara}

\goodpaper{../doctoral-consortium/pdf/EDM_2022_paper_34.pdf}{Identifying Explanations Within Student-Tutor Chat Logs}%
    {Ethan Prihar, Alexander Moore and Neil Heffernan}

\goodpaper{../doctoral-consortium/pdf/EDM_2022_paper_201.pdf}{Detecting When a Learner Requires Assistance with Programming and Delivering a Useful Hint}%
    {Marcus Messer}

\goodpaper{../doctoral-consortium/pdf/EDM_2022_paper_209.pdf}{A Paraphrase Identification Approach in Paragraph length texts}%
    {Arwa Al Saqaabi, Craig Stewart, Eleni Akrida and Alexandra Cristea}

\goodpaper{../doctoral-consortium/pdf/EDM_2022_paper_212.pdf}{Investigating learners' Cognitive Engagement in Python Programming using ICAP framework}%
    {Daevesh Singh and Ramkumar Rajendran}

\goodpaper{../doctoral-consortium/pdf/EDM_2022_paper_213.pdf}{Improving Automated Assessment and Feedback for Student Open-responses in Mathematics}%
    {Sami Baral}

\goodpaper{../doctoral-consortium/pdf/EDM_2022_paper_216.pdf}{Modeling Cognitive Load and Affect to Support Adaptive Online Learning}%
    {Minghao Cai and Carrie Demmans Epp}

\goodpaper{../doctoral-consortium/pdf/EDM_2022_paper_219.pdf}{Using AI, ML and Sentiment Analysis to Increase Diversity and Equity in Technology Training and Careers}%
    {Jonathan Young, Sue Black, Alexandra Cristea, Ryan Hodgson and Cristina Todor}

\goodpaper{../doctoral-consortium/pdf/EDM_2022_paper_225.pdf}{Effect of Q-matrix Misspecification on Variational Autoencoders (VAE) for Multidimensional Item Response Theory (MIRT) Models Estimation}%
    {Mahbubul Hasan, Lih Y Deng, John Sabatini, Dale Bowman, Ching-Chi Yang and John Hollander}

\goodpaper{../doctoral-consortium/pdf/EDM_2022_paper_228.pdf}{Towards Personalised Learning of Psychomotor Skills with Data Mining}%
    {Miguel Portaz and Olga C. Santos}

\goodpaper{../industry-track/pdf/EDM_2022_paper_5.pdf}{Using a Randomized Experiment to Compare the Performance of Two Adaptive Assessment Engines}%
    {Jeffrey Matayoshi, Hasan Uzun and Eric Cosyn}

\goodpaper{../industry-track/pdf/EDM_2022_paper_24.pdf}{Mining Artificially Generated Data to Estimate Competency}%
    {Robby Robson, Benjamin Goldberg, Shelly Blake-Plock, Cliff Casey, William Hoyt, Mike Hernandez and Fritz Ray}

\goodpaper{../industry-track/pdf/EDM_2022_paper_60.pdf}{\textquotedblleftClosing the Loop\textquotedblright in Educational Data Science with an Open Source Architecture for Large-Scale Field Trials}%
    {Stephen Fancsali, April Murphy and Steven Ritter}

\goodpaper{../industry-track/pdf/EDM_2022_paper_92.pdf}{Estimating the causal effects of Khan Academy Map Accelerator across demographic subgroups}%
    {Phillip Grimaldi, Kodi Weatherholtz and Kelli Millwood Hill}

\goodpaper{../workshop-tutorials/pdf/EDM_2022_paper_188.pdf}{The Third Workshop of The Learner Data Institute: Big Data, Research Challenges, \& Science Convergence in Educational Data Science}%
    {Vasile Rus and Stephen Fancsali}

\goodpaper{../workshop-tutorials/pdf/EDM_2022_paper_189.pdf}{FATED 2022: Fairness, Accountability, and Transparency in Educational Data}%
    {Collin Lynch, Mirko Marras, Mykola Pechenizkiy, Anna Rafferty, Steve Ritter, Vinitra Swamy and Renzhe Yu}

\goodpaper{../workshop-tutorials/pdf/EDM_2022_paper_190.pdf}{6th Educational Data Mining in Computer Science Education (CSEDM) Workshop}%
    {Bita Akram, Thomas Price, Yang Shi, Peter Brusilovsky and Sharon I-Han}

\goodpaper{../workshop-tutorials/pdf/EDM_2022_paper_191.pdf}{Rethinking Accessibility: Applications in Educational Data Mining}%
    {Juanita Hicks, Ruhan Circi, Burhan Ogut, Michelle Yin and Darrick Yee}

\goodpaper{../workshop-tutorials/pdf/EDM_2022_paper_192.pdf}{Causal Inference in Educational Data Mining}%
    {Adam Sales and Neil Heffernan}

\goodpaper{../workshop-tutorials/pdf/EDM_2022_paper_193.pdf}{Using the Open Science Framework to promote Open Science in Education Research}%
    {Stacy Shaw and Adam Sales}




%%%%%%%%%%%%%%%%%%%%%%%%%%%%%%%%%%%%%%%%%%%%%%%%%%%%%%%%%%%%%%%%%%%%%%%%%%%%%%%%%%%%%%%%%%%%%%%

\clearpage
\thispagestyle{empty}
\mbox{}
% \clearpage
% \thispagestyle{empty}
% \pagecolor{myred}
% \mbox{}

\end{document}



%%% Local Variables:
%%% mode: latex
%%% TeX-master: t
%%% End:
