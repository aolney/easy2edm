%%%% SemDial Proceedings template by Raquel Fernández, 2013.
%%%% Modified by Simon Dobnik for the proceedings of IWCS 2019.
%%%% Modified by Andrew Olney for the proceddings of EDM 2022

%TODO fancy footer https://www.overleaf.com/learn/latex/Headers_and_footers

\documentclass[letterpaper,11pt,oneside]{book} %letterpaper %a4paper
\usepackage[utf8]{inputenc} 
\usepackage[T1]{fontenc} % fonts to encode unicode
\usepackage{times}
\usepackage{pdfpages}
%\usepackage{color}
%\usepackage{calc}
\usepackage{url}
%\usepackage{xcolor}
\pagestyle{plain}


\usepackage[colorlinks,
%%% EDIT TITLE: %%%%%%%%%%%%%%%%%%%%%%%%%%%%%%%%%%%%%%%%%%%%%%%%%%%%
            pdftitle={Proceedings of the 15th International Conference on Educational Data Mining},
            pdfauthor={International Educational Data Mining Society},
            %pdfsubject={...},
            %pdfkeywords={...}
            linkcolor=blue
           ]{hyperref}   % hyperlinked table of contents, etc.

\usepackage[bottom=1in,right=1in,left=1in,top=1in]{geometry}

% Textarea
%\setlength{\textwidth}{17.7cm}
%\setlength{\textheight}{25cm}
%\setlength{\oddsidemargin}{-0.6cm}
%\setlength{\topmargin}{-1.5cm}

\renewcommand{\baselinestretch}{1.1}
\setlength{\parindent}{0pt}
\setlength{\parskip}{5pt}

\newcommand{\putframe}{}
\newcommand{\draft}{\renewcommand{\putframe}{\noindent\vspace*{-8pt}\textcolor{red}{\hrule height 1mm}
\vfill\noindent\textcolor{red}{\hrule height 1mm}}}

% Parameters: file name, title, authors, horizontal offset, vertical offset
\newcommand{\paper}[5]{%
\cleardoublepage
\phantomsection
\addcontentsline{toc}{section}{\hspace{-17pt}#2}
\addtocontents{toc}{$\,$\textit{#3}\vspace{5pt}}
\includepdf[pages=-,offset={#4 #5},pagecommand={\putframe}]{#1}
\cleardoublepage}

\newcommand{\goodpaper}[3]{\paper{#1}{#2}{#3}{-0mm}{-0mm}}


%\draft


\begin{document}
\pagenumbering{roman}
%%  TITLE PAGE 
%%%%%%%%%%%%%%%%%%%%%%%%%%%%%%%%%%%%%%%%%%%%%%%%%%%%%%%%%%%%%%%%%%%%%%%%%%%%%%%%
\pdfbookmark{Proceedings of the 15th International Conference on Educational Data Mining}{title}
\thispagestyle{empty}

\begin{center}
%  \LARGE IWCS 2019 \\
  \vspace*{55mm}
    {\bf
    %\Huge
    \LARGE
    % \fontsize{38}{46}\selectfont
    Proceedings of the 15th International Conference on \\ Educational Data Mining\\
%    \hspace*{1cm}\\ \hspace*{1cm} \\
%    \hspace*{1cm} \\ \hspace*{1cm}\\
%    \hspace*{1cm}\\
    \vspace{1cm}
    %\Huge
    \Large
    Carol Forsyth \& Stephen Fancsali (eds).
    % Proceedings of the Conference, Long Papers\\
    \vspace{5cm}
    \hspace*{1cm}} \\ % Full Volume
    %\vspace{75mm}
%    \vspace{43mm}
    \LARGE
    24--27 July, 2022\\
    University of Durham \\
    Durham, United Kingdom
  \end{center}
  
% INCLUDE SPONSOR LOGOS HERE.  Upload your images with a template set.
% Then, for a file called ``logo.png'', you would use a line like the
% the following:
%
% \includegraphics[width=2.5cm]{../templates/logo.png}
\vspace*{\fill}
\begin{center}
\includegraphics[width=4.5cm]{pics/duolingo.png} \quad \includegraphics[width=4.5cm]{pics/ets.png} \quad \includegraphics[width=4.5cm]{pics/durham.png}  %\quad \includegraphics[width=4.5cm]{pics/talkamatic.png} 
\end{center}

\clearpage

%% DETAILS 
%%%%%%%%%%%%%%%%%%%%%%%%%%%%%%%%%%%%%%%%%%%%%%%%%%%%%%%%%%%%%%%%%%%%%%%%%%%%%%%
\pdfbookmark{ISBN}{isbn}
\thispagestyle{empty}

%% INCLUDE SPONSOR LOGOS HERE.  Upload your images with a template set.
%% Then, for a file called ``logo.png'', you would use a line like the
%% the following:
%%
%% \includegraphics[width=2.5cm]{../templates/logo.png}
%\begin{center}
%\includegraphics[width=4.5cm]{pics/duolingo.png} \quad \includegraphics[width=4.5cm]{pics/ets.png} \quad \includegraphics[width=4.5cm]{pics/durham.png}  %\quad \includegraphics[width=4.5cm]{pics/talkamatic.png} 
%\end{center}

%\vspace*{\fill}
%

%\vspace*{3.5in}
\large
\noindent
\copyright 2022 International Educational Data Mining Society\\

This work is licensed under the Creative Commons Attribution-NonCommercial-NoDerivatives 4.0 International License. To view a copy of this license, visit http://creativecommons.org/licenses/by-nc-nd/4.0/ or send a letter to Creative Commons, PO Box 1866, Mountain View, CA 94042, USA.\\

Download copies of this and other EDM proceedings from:

\begin{tabular}{ll}
\ \ \ \ \ \ & International Educational Data Mining Society (IEDMS)\\
&{\tt https://educationaldatamining.org}\\
\end{tabular}\\

%\vspace*{0.6in}
\vspace*{\fill}


\textit{Proceedings of the 15th International Conference on Educational Data Mining: Industry Track}.\\ Carol Forsyth \& Stephen Fancsali (eds).\\ July 24--27, 2022. 
    Durham, United Kingdom.\\
\noindent ISBN TODO\\
%\hspace*{6.5mm} \\

%\vspace*{0.6in}
%\noindent Order copies of this and other ACL proceedings from: \\
%\vspace*{3mm}
%
%\begin{tabular}{ll}
%\ \ \ \ \ \ & Association for Computational Linguistics (ACL)\\
%& 209 N. Eighth Street\\
%& Stroudsburg, PA 18360\\
%& USA\\
%& Tel: +1-570-476-8006\\
%& Fax: +1-570-476-0860\\
%&{\tt acl@aclweb.org}\\
%\end{tabular}

%\vspace*{6mm}
%\textit{Proceedings of the 15th International Conference on Educational Data Mining: Industry Track}.\\ Carol Forsyth \& Stephen Fancsali (eds).\\ July 24--27, 2022. 
%    Durham, United Kingdom.\\
%\noindent ISBN TODO\\
% This is the ISBN for a main proceedings -- Volume 1.
% \noindent ISBN 978-1-945626-01-2 (Volume 2)\\
% This is the ISBN for  main proceedings -- Volume 2.
% Use the right ISBN for your proceedings


\clearpage

%%%%%%%%%%%%%%%%%%%%%%%%%%%%%%%%%%%%%%%%%%%%%%%%%%%%%%%%%%%%%%%%%%%%%%%%%%%%%%%%%%%%%%%%%%%%%%% 
\pdfbookmark{Preface}{preface}

%\section*{Preface}

\begin{center}
  {\Large \bf Preface}
\end{center}

\vspace*{0.5cm}

%%%%%%%%%%%%%%%%%%%%%%%%%%%%%%%%%%%%%%%%%%%%%%%%%%%%%%%%%%%%%%%%%%%%%%%%

%%% INSERT YOUR INTRO HERE
% Welcome to the ACL Workshop on Unresolved Matters. We received
% 17 submissions, and due to a rigerous review process, we rejected 16. 

Insert preface here 


\clearpage

%%%%%%%%%%%%%%%%%%%%%%%%%%%%%%%%%%%%%%%%%%%%%%%%%%%%%%%%%%%%%%%%%%%%%%%%%%%%%%%%%%%%%%%%%%%%%%% 
\pdfbookmark{Organizing Committee}{pc}

\begin{center}
  {\Large \bf Organizing Committee}
\end{center}

\vspace*{0.5cm}

%%%%%%%%%%%%%%%%%%%%%%%%%%%%%%%%%%%%%%%%%%%%%%%%%%%%%%%%%%%%%%%%%%%%%%%%

Insert organizing committee here
%\begin{description}
%% \item{\bf Organizers:}\vspace{2mm} \\
%% John Doe, Univeristy of Southern Atlantis\\
%% Jane Example, ACME Research Labs
%
%\item{\bf Organisers:}\vspace{2mm} \\
%  \emph{Local Chairs:} Stergios Chatzikyriakidis and Simon Dobnik \\
%  \emph{Program Chairs:} Stergios Chatzikyriakidis, Vera Demberg, and Simon Dobnik \\
%  \emph{Workshops Chair:} Asad Sayeed \\
%  \emph{Student Track Chairs:} Vlad Maraev and Chatrine Qwaider \\
%  \emph{Sponsorships Chair:} Staffan Larsson \\
%  
%\vspace{3mm}
%\item{\bf Program Committee:}\vspace{2mm} \\
%Lasha Abzianidze, Laura Aina, Maxime Amblard, Krasimir Angelov, Emily M. Bender, Raffaella Bernardi, Jean-Philippe   Bernardy , Rasmus Blanck, Gemma   Boleda, Alessandro   Bondielli, Lars   Borin, Johan Bos, Ellen   Breitholtz, Harry Bunt, Aljoscha  Burchardt, Nicoletta Calzolari, Emmanuele Chersoni, Philipp Cimiano, Stephen  Clark, Robin Cooper, Philippe  de Groote, Vera   Demberg, Simon  Dobnik, Devdatt   Dubhashi, Katrin  Erk, Arash   Eshghi, Raquel  Fernández, Jonathan  Ginzburg, Matthew Gotham, Eleni   Gregoromichelaki, Justyna Grudzinska, Gözde Gül Şahin, Iryna  Gurevych , Dag  Haug, Aurelie   Herbelot, Julian  Hough, Christine  Howes, Elisabetta Jezek, Richard  Johansson, Alexandre Kabbach, Lauri  Karttunen, Ruth   Kempson, Mathieu  Lafourcade, Gabriella   Lapesa, Shalom  Lappin, Staffan   Larsson, Gianluca Lebani, Kiyong  Lee, Alessandro   Lenci, Martha   Lewis, Maria Liakata, Sharid   Loáiciga, Zhaohui Luo, Moritz  Maria, Aleksandre Maskharashvili, Stephen   Mcgregor, Louise  McNally, Bruno  Mery, Mehdi  Mirzapour, Richard   Moot, Alessandro  Moschitti, Larry  Moss, Diarmuid  O Seaghdha, Sebastian   Pado, Ludovica  Pannitto, Ivandre Paraboni, Lucia C.   Passaro, Sandro   Pezzelle, Manfred Pinkal, Paul Piwek, Massimo  Poesio, Sylvain   Pogodalla, Christopher  Potts, Stephen  Pulman, Matthew   Purver, James   Pustejovsky, Alessandro   Raganato, Giulia  Rambelli, Allan   Ramsay, Aarne   Ranta, Christian  Retoré, Martin  Riedl, Roland   Roller, Mehrnoosh Sadrzadeh, Asad   Sayeed, Tatjana   Scheffler, Sabine Schulte Im Walde, Marco S. G.   Senaldi, Manfred  Stede, Matthew  Stone, Allan Third, Kees  Van  Deemter, Eva Maria  Vecchi, Carl Vogel, Ivan  Vulić, Bonnie   Webber, Roberto   Zamparelli
%
%
%\vspace{3mm}
%\item{\bf Invited Speakers:}\vspace{2mm} \\
%  % James Goodword, Academy of Hysterical Laughter
%  Mehrnoosh Sadrzadeh, Queen Mary, University of London \\
%  Ellie Pavlick,  Brown University \\
%  Raffaella Bernardi, University of Trento

% Panelists

% Invited Paper

%\end{description}


\clearpage

%%%%%%%%%%%%%%%%%%%%%%%%%%%%%%%%%%%%%%%%%%%%%%%%%%%%%%%%%%%%%%%%%%%%%%%%%%%%%%%%%%%%%%%%%%%%%%%

\pdfbookmark{Sponsors}{sponsors}

\begin{center}
  {\Large \bf Sponsors}
\end{center}

\vspace*{3cm}

\begin{center}
  {\Large \it Silver}\\
  \vspace*{0.5cm}
  \includegraphics[width=4.5cm]{pics/duolingo.png}
\end{center}

\vspace*{2cm}

\begin{center}
  {\Large \it Bronze}\\
  \vspace*{0.4cm}
  \includegraphics[width=4.5cm]{pics/ets.png} 
\end{center}


%\pdfbookmark{Invited talks}{invited}
%
%%\section*{Invited talks}
%
%\begin{center}
%  {\Large \bf Invited Talks}
%\end{center}
%
%\vspace*{0.5cm}
%
%\textbf{Mehrnoosh Sadrzadeh: Ellipsis in Compositional Distributional Semantics}
%
%Ellipsis is a natural language phenomenon where part of a sentence is missing and its information must be recovered from its surrounding context, as in ``Cats chase dogs and so do foxes.''. Formal semantics offers different methods for resolving ellipsis and recovering the missing information, but the problem has not been considered for distributional semantics, where words have vector embeddings and combinations thereof provide embeddings for sentences. In elliptical sentences these combinations go beyond linear as copying of elided information is necessary. I will talk about recent results in our NAACL 2019 paper, joint with G. Wijnholds, where we develop different models for embedding VP-elliptical sentences using modal sub-exponential categorial grammars. We extend existing verb disambiguation and sentence similarity datasets to ones containing elliptical phrases and evaluate our models on these datasets for a variety of linear and non-linear combinations. Our results show that indeed resolving ellipsis improves the performance of vectors and tensors on these tasks and it also sheds some light on disambiguating their sloppy and strict  readings.
%
%\bigskip
%
%\textbf{Ellie Pavlick: What Should Constitute Natural Language ``understanding''?}
%
%Natural language processing has become indisputably good over the past few years. We can perform retrieval and question answering with purported super-human accuracy, and can generate full documents of text that seem good enough to pass the Turing test. In light of these successes, it is tempting to attribute the empirical performance to a deeper "understanding" of language that the models have acquired. Measuring natural language "understanding", however, is itself an unsolved research problem. In this talk, I will discuss recent work which attempts to illuminate what it is that state-of-the-art models of language are capturing. I will describe approaches which evaluate the models' inferential behaviour, as well as approaches which rely on inspecting the models' internal structure directly. I will conclude with results on human's linguistic inferences, which highlight the challenges involved with developing prescriptivist language tasks for evaluating computational models. 
%
%\bigskip
%
%\textbf{Raffaella Bernardi: Beyond Task Success: A Closer Look at Jointly Learning to See, Ask,
%and GuessWhat}
%
%The development of conversational agents that ground language into visual information is a challenging problem that requires the integration of dialogue management skills with multimodal understanding. Recently, visual dialogue settings have entered the scene of the Machine Learning and Computer Vision communities thanks to the construction of visually grounded human-human dialogue datasets against which Neural Network models (NNs) have been challenged. I will present our work on GuessWhat?! in which two NN agents interact to each other so that one of the two (the Questioner), by asking questions to the other (the Answerer), can guess which object the Answerer has in mind among all the entities in a given image (GuessWhat?!).  I will present our Questioner model: it encodes both visual and textual inputs, produces a multimodal representation, generates natural language questions, understands the Answerers' responses and guesses the object. I will compare our model's dialogues with models that exploit much more complex learning paradigms, like Reinforcement Learning, showing that more complex machine learning methods do not necessarily correspond to better dialogue quality or even better quantitative performance. The talk is based on work available at \url{https://vista-unitn-uva.github.io/}.


\clearpage

%%%%%%%%%%%%%%%%%%%%%%%%%%%%%%%%%%%%%%%%%%%%%%%%%%%%%%%%%%%%%%%%%%%%%%%%%%%%%%%%%%%%%%%%%%%%%%%


\pdfbookmark{Table of Contents}{toc}
\setlength{\parskip}{0pt}

\renewcommand{\contentsname}{\mbox{}\\[-108pt]\noindent\textbf{\Large
    Table of Contents}\\[-28pt]}
\tableofcontents
\cleardoublepage

\pagenumbering{arabic}
%%%%%%%%%%%%%%%%%%%%%%%%%%%%%%%%%%%%%%%%%%%%%%%%%%%%%%%%%%%%%%%%%%%%%%%%%%%%%%%%%%%%%%%%%%%%%%%
% \pdfbookmark{Invited Talks}{invited}
% \thispagestyle{empty}
% \mbox{}\vfill
% \begin{center}
% \Huge \bf Invited Talks
% \end{center}
% \mbox{}\vfill

% \clearpage

% \addtocontents{toc}{\vspace{10pt} $\,$\textbf{Invited Talks}\vspace{5pt}}
 


% \pdfbookmark{Full Papers}{papers}
% \thispagestyle{empty}
% \mbox{}\vfill
% \begin{center}
% \Huge \bf Full Papers
% \end{center}
% \mbox{}\vfill

% \clearpage

% \addtocontents{toc}{{}\\[10pt] \textbf{Full Papers}\vspace{5pt}}



% \pdfbookmark{Poster Abstracts}{posters}
% \thispagestyle{empty}
% \mbox{}\vfill
% \begin{center}
% \Huge \bf Poster Abstracts
% \end{center}
% \mbox{}\vfill

% \clearpage

% \addtocontents{toc}{{}\\[10pt] \textbf{Poster Abstracts}\vspace{5pt}}

% \goodpaper{../pdf/IWCS_2019_paper_1.pdf}{Temporal and Aspectual Entailment}%
% {Thomas Kober, Sander Bijl de Vroe and Mark Steedman}

% \goodpaper{../pdf/IWCS_2019_paper_3.pdf}{Re-Ranking Words to Improve Interpretability of Automatically Generated Topics}%
% {Areej Alokaili, Nikolaos Aletras and Mark Stevenson}

\goodpaper{../long-papers/pdf/EDM_2022_paper_7.pdf}{Insta-Reviewer: A Data-Driven Approach for Generating Instant Feedback on Students' Project Reports}%
    {Qinjin Jia, Mitchell Young, Yunkai Xiao, Jialin Cui, Chengyuan Liu, Parvez Rashid and Edward Gehringer}

\goodpaper{../long-papers/pdf/EDM_2022_paper_10.pdf}{Sparse Factor Autoencoders for Item Response Theory}%
    {Benjamin Paa\ssen, Malwina Dywel, Melanie Fleckenstein and Niels Pinkwart}

\goodpaper{../long-papers/pdf/EDM_2022_paper_12.pdf}{Exploring Common Trends in Online Educational Experiments}%
    {Ethan Prihar, Manaal Syed, Korinn Ostrow, Stacy Shaw, Adam Sales and Neil Heffernan}

\goodpaper{../long-papers/pdf/EDM_2022_paper_18.pdf}{Designing Representations for Question Sequencing using Reinforcement Learning}%
    {Aqil Zainal Azhar, Avi Segal and Kobi Gal}

\goodpaper{../long-papers/pdf/EDM_2022_paper_21.pdf}{Code-DKT: A Code-based Knowledge Tracing Model for Programming Tasks}%
    {Yang Shi, Min Chi, Tiffany Barnes and Thomas Price}

\goodpaper{../long-papers/pdf/EDM_2022_paper_22.pdf}{Building a Reinforcement Learning Environment from Limited Data to Optimize Teachable Robot Interventions}%
    {Tristan Maidment, Mingzhi Yu, Nikki Lobczowski, Adriana Kovashka, Erin Walker, Diane Litman and Timothy Nokes-Malach}

\goodpaper{../long-papers/pdf/EDM_2022_paper_35.pdf}{Detecting SMART Model Cognitive Operations in Mathematical Problem-Solving Process}%
    {Jiayi Zhang, Juliana Ma. Alexandra L. Andres, Stephen Hutt, Ryan S. Baker, Jaclyn Ocumpaugh, Caitlin Mills, Jamiella Brooks, Sheela Sethuraman and Tyron Young}

\goodpaper{../long-papers/pdf/EDM_2022_paper_37.pdf}{SQL-DP:A Novel Difficulty Prediction Framework for SQL Programming Problems}%
    {Jia Xu, Tingting Wei and Pin Lv}

\goodpaper{../long-papers/pdf/EDM_2022_paper_51.pdf}{Evaluating the Explainers: Black-Box Explainable Machine Learning for Student Success Prediction in MOOCs}%
    {Vinitra Swamy, Bahar Radmehr, Natasa Krco, Mirko Marras and Tanja K\"{a}ser}

\goodpaper{../long-papers/pdf/EDM_2022_paper_53.pdf}{Addressing Competing Objectives in Allocating Funds to Scholarships and Need-based Financial Aid}%
    {Vinthuy Phan, Laura Wright and Bridgette Decent}

\goodpaper{../long-papers/pdf/EDM_2022_paper_54.pdf}{Automatic Short Math Answer Grading via In-context Meta-learning}%
    {Mengxue Zhang, Sami Baral, Neil Heffernan and Andrew Lan}

\goodpaper{../long-papers/pdf/EDM_2022_paper_55.pdf}{Investigating Multimodal Predictors of Peer Satisfaction for Collaborative Coding in Middle School}%
    {Yingbo Ma, Gloria Ashiya Katuka, Mehmet Celepkolu and Kristy Elizabeth Boyer}

\goodpaper{../long-papers/pdf/EDM_2022_paper_56.pdf}{Going Deep and Far: Gaze-based Models Predict Multiple Depths of Comprehension During and One Week Following Reading}%
    {Megan Caruso, Candace Peacock, Rosy Southwell, Guojing Zhou and Sidney D'Mello}

\goodpaper{../long-papers/pdf/EDM_2022_paper_62.pdf}{Predicting Reading Comprehension Scores of Elementary School Students}%
    {Yuyang Nie, Helene Deacon, Alona Fyshe and Carrie Demmans Epp}

\goodpaper{../long-papers/pdf/EDM_2022_paper_63.pdf}{Enhancing Stealth Assessment in Game-Based Learning Environments with Generative Zero-Shot Learning}%
    {Nathan Henderson, Halim Acosta, Wookhee Min, Bradford Mott, Trudi Lord, Frieda Reichsman, Chad Dorsey, Eric Wiebe and James Lester}

\goodpaper{../long-papers/pdf/EDM_2022_paper_78.pdf}{Generalisable Methods for Early Prediction in Interactive Simulations for Education}%
    {Jade Ma\"{\i} Cock, Mirko Marras, Christian Giang and Tanja K\"{a}ser}

\goodpaper{../long-papers/pdf/EDM_2022_paper_80.pdf}{Toward Better Grade Prediction via A2GP - An Academic Achievement Inspired Predictive Model}%
    {Wei Qiu, S. Supraja and Andy W. H. Khong}

\goodpaper{../long-papers/pdf/EDM_2022_paper_84.pdf}{Individual Fairness Evaluation for Automated Essay Scoring System}%
    {Afrizal Doewes, Akrati Saxena, Yulong Pei and Mykola Pechenizkiy}

\goodpaper{../long-papers/pdf/EDM_2022_paper_94.pdf}{Combining domain modelling and student modelling techniques in a single automated pipeline}%
    {Gio Picones, Benjamin Paa\ssen, Irena Koprinska and Kalina Yacef}

\goodpaper{../long-papers/pdf/EDM_2022_paper_108.pdf}{Neural Recall Network: A Neural Network Solution to Low Recall Problem in Regex-based Qualitative Coding}%
    {Zhiqiang Cai, Cody Marquart and David Shaffer}

\goodpaper{../long-papers/pdf/EDM_2022_paper_109.pdf}{Towards Including Instructor Features in Student Grade Prediction}%
    {Nathan Ong, Jiaye Zhu and Daniel Mosse}

\goodpaper{../long-papers/pdf/EDM_2022_paper_125.pdf}{Item Response Theory-Based Gaming Detection}%
    {Yun Huang, Steven Dang, J. Elizabeth Richey, Michael Asher, Nikki G. Lobczowski, Danielle Chine, Elizabeth A. McLaughlin, Judith M. Harackiewicz, Vincent Aleven and Kenneth Koedinger}

\goodpaper{../long-papers/pdf/EDM_2022_paper_133.pdf}{Exploring Cultural Diversity and Collaborative Team Communication through a Dynamical Systems Lens}%
    {Mohammad Amin Samadi, Jacqueline G. Cavazos, Yiwen Lin and Nia Nixon}

\goodpaper{../long-papers/pdf/EDM_2022_paper_141.pdf}{Predicting Cognitive Engagement in Online Course Discussion Forums}%
    {Guher Gorgun, Seyma Nur Yildirim-Erbasli and Carrie Demmans Epp}

\goodpaper{../long-papers/pdf/EDM_2022_paper_148.pdf}{Investigating Temporal Dynamics Underlying Successful Collaborative Problem Solving Behaviors with Multilevel Vector Autoregression}%
    {Guojing Zhou, Robert Moulder, Chen Sun and Sidney D'Mello}

\goodpaper{../long-papers/pdf/EDM_2022_paper_155.pdf}{Challenges and Feasibility of Automatic Speech Recognition for Modeling Student Collaborative Discourse in Classrooms}%
    {Rosy Southwell, Samuel Pugh, E. Margaret Perkoff, Charis Clevenger, Jeffrey Bush, Rachel Lieber, Wayne Ward, Peter Foltz and Sidney D'Mello}

\goodpaper{../long-papers/pdf/EDM_2022_paper_3.pdf}{Does Practice Make Perfect? Analyzing the Relationship Between Higher Mastery and Forgetting in an Adaptive Learning System}%
    {Jeffrey Matayoshi, Eric Cosyn and Hasan Uzun}

\goodpaper{../short-papers/pdf/EDM_2022_paper_9.pdf}{Improving Peer Assessment with Graph Neural Networks}%
    {Alireza A. Namanloo, Julie Thorpe and Amirali Salehi-Abari}

\goodpaper{../short-papers/pdf/EDM_2022_paper_14.pdf}{Using Neural Network-Based Knowledge Tracing for a Learning System with Unreliable Skill Tags}%
    {Shamya Karumbaiah, Jiayi Zhang, Ryan Baker, Richard Scruggs, Whitney Cade, Margaret Clements and Shuqiong Lin}

\goodpaper{../long-papers/pdf/EDM_2022_paper_16.pdf}{\#lets-discuss: Analyzing Student Affect in Course Forums Using Emoji}%
    {Ariel Blobstein, Kobi Gal, David Karger, Marc Facciotti, Hyunsoo Kim, Jumana Almahmoud and Kamali Sripathi}

\goodpaper{../long-papers/pdf/EDM_2022_paper_20.pdf}{Investigating Growth of Representational Competencies by Knowledge-Component Model}%
    {Jihyun Rho, Martina Rau and Barry Vanveen}

\goodpaper{../short-papers/pdf/EDM_2022_paper_31.pdf}{Adversarial bandits for drawing generalizable conclusions in non-adversarial experiments: an empirical study}%
    {Yang Zhi-Han, Shiyue Zhang and Anna Rafferty}

\goodpaper{../short-papers/pdf/EDM_2022_paper_40.pdf}{Towards Real Interpretability of Student Success Prediction Combining Methods of XAI and Social Science}%
    {Lea Cohausz}

\goodpaper{../short-papers/pdf/EDM_2022_paper_42.pdf}{An Evaluation of code2vec Embeddings for Scratch}%
    {Benedikt Fein, Isabella Gra\ssl, Florian Beck and Gordon Fraser}

\goodpaper{../long-papers/pdf/EDM_2022_paper_43.pdf}{Investigating the effect of Automated Feedback on learning behavior in MOOCs for programming}%
    {Hagit Gabbay and Anat Cohen}

\goodpaper{../short-papers/pdf/EDM_2022_paper_45.pdf}{Data-driven goal setting: Searching optimal badges in the decision forest}%
    {Julian Langenhagen}

\goodpaper{../short-papers/pdf/EDM_2022_paper_52.pdf}{Improving problem detection in peer assessment through pseudo-labeling using semi-supervised learning}%
    {Chengyuan Liu, Jialin Cui, Ruixuan Shang, Yunkai Xiao, Qinjin Jia and Edward Gehringer}

\goodpaper{../long-papers/pdf/EDM_2022_paper_59.pdf}{Evaluating Gaming Detector Models For Robustness Over Time}%
    {Nathan Levin, Ryan Baker, Nidhi Nasiar, Stephen Fancsali and Stephen Hutt}

\goodpaper{../long-papers/pdf/EDM_2022_paper_69.pdf}{Characterizing joint attention dynamics during collaborative problem-solving in an immersive astronomy simulation}%
    {Yiqiu Zhou and Jina Kang}

\goodpaper{../long-papers/pdf/EDM_2022_paper_72.pdf}{Can Population-based Engagement improve Personalisation? A Novel Dataset, Baselines and Experiments}%
    {Sahan Bulathwela, Meghana Verma, Mar\'{\i}a P\'{e}rez Ortiz, Emine Yilmaz and John Shawe-Taylor}

\goodpaper{../long-papers/pdf/EDM_2022_paper_74.pdf}{Is there Method in Your Mistakes? Capturing Error Contexts by Graph Mining for Targeted Feedback}%
    {Maximilian Jahnke and Frank H\"{o}ppner}

\goodpaper{../short-papers/pdf/EDM_2022_paper_82.pdf}{Log mining for course recommendation in limited information scenarios}%
    {Juan Sanguino, Ruben Manrique, Olga Mari\~{n}o, Mario Linares and Nicolas Cardozo}

\goodpaper{../long-papers/pdf/EDM_2022_paper_93.pdf}{Using Machine Learning Explainability Methods to Personalize Interventions for Students}%
    {Paul Hur, HaeJin Lee, Suma Bhat and Nigel Bosch}

\goodpaper{../long-papers/pdf/EDM_2022_paper_97.pdf}{Grade Prediction via Prior Grades and Text Mining on Course Descriptions: Course Outlines and Intended Learning Outcomes}%
    {Jiawei Li, S. Supraja, Wei Qiu and Andy W. H. Khong}

\goodpaper{../long-papers/pdf/EDM_2022_paper_105.pdf}{From \lbraceSolution\rbrace Synthesis to \lbraceStudent Attempt\rbrace Synthesis for Block-Based Visual Programming Tasks}%
    {Adish Singla and Nikitas Theodoropoulos}

\goodpaper{../short-papers/pdf/EDM_2022_paper_107.pdf}{Mining Assignment Submission Time to Detect At-Risk Students with Peer Information}%
    {Yuancheng Wang, Nanyu Luo and Jianjun Zhou}

\goodpaper{../short-papers/pdf/EDM_2022_paper_118.pdf}{Simulating Policy Changes in Prerequisite-Free Curricula: A Supervised Data-Driven Approach}%
    {Frederik Baucks and Laurenz Wiskott}

\goodpaper{../short-papers/pdf/EDM_2022_paper_120.pdf}{Modeling One-on-one Online Tutoring Discourse using an Accountable Talk Framework}%
    {Renu Balyan, Tracy Arner, Karen Taylor, Jinnie Shin, Michelle Banawan, Walter Leite and Danielle McNamara}

\goodpaper{../short-papers/pdf/EDM_2022_paper_126.pdf}{Using Markov Models and Random Walks to Examine Strategy Use of More or Less Successful Comprehenders}%
    {Katerina Christhilf, Natalie Newton, Reese Butterfuss, Kathryn S. McCarthy, Laura K. Allen, Joseph P. Magliano and Danielle S. McNamara}

\goodpaper{../long-papers/pdf/EDM_2022_paper_130.pdf}{No Meaning Left Unlearned: Predicting Learners' Knowledge of Atypical Meanings of Words from Vocabulary Tests for Their Typical Meanings}%
    {Yo Ehara}

\goodpaper{../short-papers/pdf/EDM_2022_paper_145.pdf}{Using community-based problems to increase motivation in a data science virtual internship}%
    {Jillian Johnson and Andrew Olney}

\goodpaper{../short-papers/pdf/EDM_2022_paper_151.pdf}{Admitting you have a problem is the first step: Modeling when and why students seek help in programming assignments.}%
    {Zhikai Gao, Bradley Erickson, Yiqiao Xu, Collin Lynch, Sarah Heckman and Tiffany Barnes}

\goodpaper{../short-papers/pdf/EDM_2022_paper_153.pdf}{Going beyond \textquotedblleftGood job\textquotedblright: Analyzing helpful feedback from students' perspectives.}%
    {M Parvez Rashid, Yunkai Xiao and Edward F. Gehringer}

\goodpaper{../short-papers/pdf/EDM_2022_paper_168.pdf}{The AI Teacher Test: Measuring the Pedagogical Ability of Blender and GPT-3 in Educational Dialogues}%
    {Ana\"{\i}s Tack and Chris Piech}

\goodpaper{../short-papers/pdf/EDM_2022_paper_195.pdf}{Automatic Classification of Learning Objectives Based on Bloom's Taxonomy}%
    {Yuheng Li, Mladen Rakovic, Boon Xin Poh, Dragan Gasevic and Guanliang Chen}

\goodpaper{../long-papers/pdf/EDM_2022_paper_6.pdf}{Skills Taught vs Skills Sought: Using Skills Analytics to Identify the Gaps between Curriculum and Job Markets}%
    {Alireza Ahadi, Kirsty Kitto, Marian-Andrei Rizoiu and Katarzyna Musial}

\goodpaper{../long-papers/pdf/EDM_2022_paper_15.pdf}{DeepIRT with a Hypernetwork to Optimize the Degree of Forgetting of Past Data}%
    {Emiko Tsutsumi, Yiming Guo and Maomi Ueno}

\goodpaper{../short-papers/pdf/EDM_2022_paper_17.pdf}{Mining and Assessing Anomalies in Students' Online Learning Activities with Self-supervised Machine Learning}%
    {Lan Jiang and Nigel Bosch}

\goodpaper{../short-papers/pdf/EDM_2022_paper_23.pdf}{Faster Confidence Intervals for Item Response Theory via an Approximate Likelihood Profile}%
    {Benjamin Paa\ssen, Christina G\"{o}pfert and Niels Pinkwart}

\goodpaper{../long-papers/pdf/EDM_2022_paper_26.pdf}{Towards the understanding of cultural differences in between gamification preferences: A data-driven comparison between the US and Brazil}%
    {Armando Toda, Ana Klock, Filipe Dwan Pereira, Luiz Antonio Rodrigues, Paula Toledo Palomino, Vinicius Lopes, Craig Stewart, Elaine H. T. Oliveira, Isabela Gasparini, Seiji Isotani and Alexandra Cristea}

\goodpaper{../short-papers/pdf/EDM_2022_paper_33.pdf}{Supervised Machine Learning for Modelling STEM Career and Education Interest in Irish School Children}%
    {Annika Lindh, Keith Quille, Aidan Mooney, Kevin Marshall and Katriona O'Sullivan}

\goodpaper{../short-papers/pdf/EDM_2022_paper_39.pdf}{Equitable Ability Estimation in Neurodivergent Student Populations with Zero-Inflated Learner Models}%
    {Niall Twomey, Sarah McMullan, Anat Elhalal, Rafael Poyiadzi and Luis Vaquero}

\goodpaper{../short-papers/pdf/EDM_2022_paper_48.pdf}{Equity and Fairness of Bayesian Knowledge Tracing}%
    {Sebastian Tschiatschek, Maria Knobelsdorf and Adish Singla}

\goodpaper{../short-papers/pdf/EDM_2022_paper_57.pdf}{Analyzing the Equity of the Brazilian National High School Exam by Validating the Item Response Theory's Invariance}%
    {Vitoria Guardieiro, Marcos M. Raimundo and Jorge Poco}

\goodpaper{../long-papers/pdf/EDM_2022_paper_61.pdf}{A deep dive into microphone hardware for recording collaborative group work}%
    {Mariah Bradford, Paige Hansen, J. Ross Beveridge, Nikhil Krishnaswamy and Nathaniel Blanchard}

\goodpaper{../short-papers/pdf/EDM_2022_paper_67.pdf}{Linguistic Profiles, Question Framing and Performance in a Conversation-based Assessment System}%
    {Carol Forsyth, Jesse Sparks, Jonathan Steinberg and Laura McCulla}

\goodpaper{../long-papers/pdf/EDM_2022_paper_71.pdf}{Does chronology matter? Sequential vs contextual approaches to knowledge tracing}%
    {Yueqi Wang and Zachary Pardos}

\goodpaper{../short-papers/pdf/EDM_2022_paper_86.pdf}{Algorithmic unfairness mitigation in student models: When fairer methods lead to unintended results}%
    {Frank Stinar and Nigel Bosch}

\goodpaper{../short-papers/pdf/EDM_2022_paper_87.pdf}{The Impact of Semester Gaps on Student Grades}%
    {Gary Weiss, Joseph Denham and Daniel Leeds}

\goodpaper{../long-papers/pdf/EDM_2022_paper_88.pdf}{Assessing Instructor Effectiveness Based on Future Student Performance}%
    {Gary Weiss, Erik Brown, Michael Riad-Zaky, Ruby Iannone and Daniel Leeds}

\goodpaper{../short-papers/pdf/EDM_2022_paper_90.pdf}{Modeling study duration considering course enrollments and student diversity}%
    {Niels Seidel}

\goodpaper{../long-papers/pdf/EDM_2022_paper_91.pdf}{Generalized Sequential Pattern Mining of Undergraduate Courses}%
    {Daniel Leeds, Cody Chen, Yijun Zhao, Fiza Metla, James Guest and Gary Weiss}

\goodpaper{../short-papers/pdf/EDM_2022_paper_101.pdf}{Employing Tree-based Algorithms to Predict Students' Self-Efficacy in PISA 2018}%
    {Bin Tan and Maria Cutumisu}

\goodpaper{../short-papers/pdf/EDM_2022_paper_102.pdf}{Identifying Longitudinal Attendance Patterns through Student Subpopulation Distribution Comparison}%
    {Zitong Zhao, Pan Deng and Jianjun Zhou}

\goodpaper{../short-papers/pdf/EDM_2022_paper_103.pdf}{Improved Automated Essay Scoring using Gaussian Multi-Class SMOTE for Dataset Sampling}%
    {Jih Soong Tan, Ian K. T. Tan, Lay Ki Soon and Huey Fang Ong}

\goodpaper{../short-papers/pdf/EDM_2022_paper_110.pdf}{A Topic-Centric Crowdsourced Assisted Biomedical Literature Review Framework for Academics}%
    {Ryan Hodgson, Jingyun Wang, Alexandra Cristea, Fumiko Matsuzaki and Hiroyuki Kubota}

\goodpaper{../long-papers/pdf/EDM_2022_paper_115.pdf}{Personalized and Explainable Course Recommendations for Students at Risk of Dropping out}%
    {Kerstin Wagner, Agathe Merceron, Petra Sauer and Niels Pinkwart}

\goodpaper{../short-papers/pdf/EDM_2022_paper_119.pdf}{A deep reinforcement learning approach to automatic formative feedback}%
    {Aubrey Condor and Zachary Pardos}

\goodpaper{../short-papers/pdf/EDM_2022_paper_124.pdf}{Preliminary Experiments with Transformer based Approaches To Automatically Inferring Domain Models from Textbooks}%
    {Rabin Banjade, Priti Oli, Lasang Jimba Tamang and Vasile Rus}

\goodpaper{../long-papers/pdf/EDM_2022_paper_132.pdf}{E-learning Preparedness: A Key Consideration to Promote Fair Learning Analytics Development in Higher Education}%
    {Jinnie Shin, Okan Bulut and Wallace N. Pinto Jr.}

\goodpaper{../long-papers/pdf/EDM_2022_paper_136.pdf}{Leveraging Auxiliary Data from Similar Problems to Improve Automatic Open Response Scoring}%
    {Raysa Rivera-Bergollo, Sami Baral, Anthony Botelho and Neil Heffernan}

\goodpaper{../short-papers/pdf/EDM_2022_paper_138.pdf}{Modifying Deep Knowledge Tracing for Multi-step Problems}%
    {Qiao Zhang, Zeyu Chen, Natasha Lalwani and Christopher MacLellan}

\goodpaper{../long-papers/pdf/EDM_2022_paper_157.pdf}{Clustering Students Using Pre-Midterm Behaviour Data and Predict Their Exam Performance}%
    {Huanyi Chen and Paul Ward}

\goodpaper{../short-papers/pdf/EDM_2022_paper_162.pdf}{Format-Aware Item Response Theory for Predicting Vocabulary Proficiency}%
    {Boxuan Ma, Gayan Prasad Hettiarachchi and Yuji Ando}

\goodpaper{../short-papers/pdf/EDM_2022_paper_167.pdf}{Towards Generalized Methods for Automatic Question Generation in Educational Domains}%
    {Shravya Bhat, Huy Nguyen, Steven Moore, John Stamper, Majd Sakr and Eric Nyberg}

\goodpaper{../long-papers/pdf/EDM_2022_paper_175.pdf}{MOOC-Rec: Instructional Video Clip Recommendation for MOOC Forum Questions}%
    {Peide Zhu, Claudia Hauff and Jie Yang}

\goodpaper{../long-papers/pdf/EDM_2022_paper_186.pdf}{Automatical Graph-based Knowledge Tracing}%
    {Ting Long, Yunfei Liu, Weinan Zhang, Wei Xia, Zhicheng He, Ruiming Tang and Yong Yu}

\goodpaper{../long-papers/pdf/EDM_2022_paper_187.pdf}{Process-BERT: A Framework for Representation Learning on Educational Process Data}%
    {Alexander Scarlatos, Christopher Brinton and Andrew Lan}

\goodpaper{../short-papers/pdf/EDM_2022_paper_196.pdf}{Online Item Response Theory (OIRT) - Tracking Student Abilities in Online Learning System}%
    {Luyao Peng and Chengzhi Wei}

\goodpaper{../posters/pdf/EDM_2022_paper_200.pdf}{Looking for the best data fusion model in Smart Learning Environments for detecting at risk university students}%
    {Cristobal Romero, Wilson Chango and Rebeca Cerezo}

\goodpaper{../posters/pdf/EDM_2022_paper_202.pdf}{Distance measure between instructor-recommended and learner's learning pathways}%
    {Marie-Luce Bourguet and Yushan Li}

\goodpaper{../posters/pdf/EDM_2022_paper_205.pdf}{Student Perception on the Effectiveness of On-Demand Assistance in Online Learning Platforms}%
    {Aaron Haim and Neil Heffernan}

\goodpaper{../posters/pdf/EDM_2022_paper_211.pdf}{Theory-Informed Problem-Solving Sequential Pattern Vis-ualization}%
    {Zilong Pan and Min Liu}

\goodpaper{../posters/pdf/EDM_2022_paper_214.pdf}{CHUNK Learning: A Tool that Supports Personalized Education}%
    {Ralucca Gera, D'Marie Bartolf, Simona Tick and Akrati Saxena}

\goodpaper{../posters/pdf/EDM_2022_paper_215.pdf}{Do College Students Learn Math the Same Way as Middle School Students? On the Transferability of Findings on Within-Problem Supports in Intelligent Tutoring Systems}%
    {Michael Smalenberger and Kelly Smalenberger}

\goodpaper{../posters/pdf/EDM_2022_paper_220.pdf}{Comparison of Learning Behaviors on an e-Book System in 2019 Onsite and 2020 Online Courses}%
    {Hiroaki Kawashima}

\goodpaper{../posters/pdf/EDM_2022_paper_221.pdf}{Recommendation System of Mobile Language Learning Applications: Similarity versus Diversity in Learner Preference}%
    {Juyeong Song, Kisu Yang, Hyeji Jang and Hyo-Jeong So}

\goodpaper{../posters/pdf/EDM_2022_paper_222.pdf}{A Variant of Performance Factors Analysis Model for Categorization}%
    {Meng Cao and Philip Pavlik}

\goodpaper{../posters/pdf/EDM_2022_paper_223.pdf}{Selecting Reading Texts Suitable for Incidental Vocabulary Learning by Considering the Estimated Distribution of Acquired Vocabulary}%
    {Yo Ehara}

\goodpaper{../doctoral-consortium/pdf/EDM_2022_paper_34.pdf}{Identifying Explanations Within Student-Tutor Chat Logs}%
    {Ethan Prihar, Alexander Moore and Neil Heffernan}

\goodpaper{../doctoral-consortium/pdf/EDM_2022_paper_201.pdf}{Detecting When a Learner Requires Assistance with Programming and Delivering a Useful Hint}%
    {Marcus Messer}

\goodpaper{../doctoral-consortium/pdf/EDM_2022_paper_209.pdf}{A Paraphrase Identification Approach in Paragraph length texts}%
    {Arwa Al Saqaabi, Craig Stewart, Eleni Akrida and Alexandra Cristea}

\goodpaper{../doctoral-consortium/pdf/EDM_2022_paper_212.pdf}{Investigating learners' Cognitive Engagement in Python Programming using ICAP framework}%
    {Daevesh Singh and Ramkumar Rajendran}

\goodpaper{../doctoral-consortium/pdf/EDM_2022_paper_213.pdf}{Improving Automated Assessment and Feedback for Student Open-responses in Mathematics}%
    {Sami Baral}

\goodpaper{../doctoral-consortium/pdf/EDM_2022_paper_216.pdf}{Modeling Cognitive Load and Affect to Support Adaptive Online Learning}%
    {Minghao Cai and Carrie Demmans Epp}

\goodpaper{../doctoral-consortium/pdf/EDM_2022_paper_219.pdf}{Using AI, ML and Sentiment Analysis to Increase Diversity and Equity in Technology Training and Careers}%
    {Jonathan Young, Sue Black, Alexandra Cristea, Ryan Hodgson and Cristina Todor}

\goodpaper{../doctoral-consortium/pdf/EDM_2022_paper_225.pdf}{Effect of Q-matrix Misspecification on Variational Autoencoders (VAE) for Multidimensional Item Response Theory (MIRT) Models Estimation}%
    {Mahbubul Hasan, Lih Y Deng, John Sabatini, Dale Bowman, Ching-Chi Yang and John Hollander}

\goodpaper{../doctoral-consortium/pdf/EDM_2022_paper_228.pdf}{Towards Personalised Learning of Psychomotor Skills with Data Mining}%
    {Miguel Portaz and Olga C. Santos}

\goodpaper{../industry-track/pdf/EDM_2022_paper_5.pdf}{Using a Randomized Experiment to Compare the Performance of Two Adaptive Assessment Engines}%
    {Jeffrey Matayoshi, Hasan Uzun and Eric Cosyn}

\goodpaper{../industry-track/pdf/EDM_2022_paper_24.pdf}{Mining Artificially Generated Data to Estimate Competency}%
    {Robby Robson, Benjamin Goldberg, Shelly Blake-Plock, Cliff Casey, William Hoyt, Mike Hernandez and Fritz Ray}

\goodpaper{../industry-track/pdf/EDM_2022_paper_60.pdf}{\textquotedblleftClosing the Loop\textquotedblright in Educational Data Science with an Open Source Architecture for Large-Scale Field Trials}%
    {Stephen Fancsali, April Murphy and Steven Ritter}

\goodpaper{../industry-track/pdf/EDM_2022_paper_92.pdf}{Estimating the causal effects of Khan Academy Map Accelerator across demographic subgroups}%
    {Phillip Grimaldi, Kodi Weatherholtz and Kelli Millwood Hill}

\goodpaper{../workshop-tutorials/pdf/EDM_2022_paper_188.pdf}{The Third Workshop of The Learner Data Institute: Big Data, Research Challenges, \& Science Convergence in Educational Data Science}%
    {Vasile Rus and Stephen Fancsali}

\goodpaper{../workshop-tutorials/pdf/EDM_2022_paper_189.pdf}{FATED 2022: Fairness, Accountability, and Transparency in Educational Data}%
    {Collin Lynch, Mirko Marras, Mykola Pechenizkiy, Anna Rafferty, Steve Ritter, Vinitra Swamy and Renzhe Yu}

\goodpaper{../workshop-tutorials/pdf/EDM_2022_paper_190.pdf}{6th Educational Data Mining in Computer Science Education (CSEDM) Workshop}%
    {Bita Akram, Thomas Price, Yang Shi, Peter Brusilovsky and Sharon I-Han}

\goodpaper{../workshop-tutorials/pdf/EDM_2022_paper_191.pdf}{Rethinking Accessibility: Applications in Educational Data Mining}%
    {Juanita Hicks, Ruhan Circi, Burhan Ogut, Michelle Yin and Darrick Yee}

\goodpaper{../workshop-tutorials/pdf/EDM_2022_paper_192.pdf}{Causal Inference in Educational Data Mining}%
    {Adam Sales and Neil Heffernan}

\goodpaper{../workshop-tutorials/pdf/EDM_2022_paper_193.pdf}{Using the Open Science Framework to promote Open Science in Education Research}%
    {Stacy Shaw and Adam Sales}




%%%%%%%%%%%%%%%%%%%%%%%%%%%%%%%%%%%%%%%%%%%%%%%%%%%%%%%%%%%%%%%%%%%%%%%%%%%%%%%%%%%%%%%%%%%%%%%

\clearpage
\thispagestyle{empty}
\mbox{}
% \clearpage
% \thispagestyle{empty}
% \pagecolor{myred}
% \mbox{}

\end{document}



%%% Local Variables:
%%% mode: latex
%%% TeX-master: t
%%% End:
